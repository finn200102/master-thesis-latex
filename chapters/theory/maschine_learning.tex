
\section{Maschine Learning}
\label{sec:maschinelearning}
\subsection{Regression}
Für Regressionsaufgaben wird zwischen einer Antwortvariable \(Y\) und einer erklärenden Varaible \(X\) bzw. mehreren ein Zusammenhang hergestellt mit einem Model dieser Daten. Dieses Model wird durch Parameter spezifizierte, diese zu schätzen ist Aufgabe der Regression. Um den Fehler des Models festzustellen wird die Methode der kleinsten Quadrate verwendet, dabei werden die Parameter so gewählt, dass diese die 'residuar sum of squares' (RSS) minimieren. Neben der RSS existieren weiter Fehlermaße die von Nutze sind, bei der Bewertung von Regressionsmodeelen.

\subsubsection{Mean Squared Error (MSE)}
Für die Optimierung von den meißten Regressionsmodellen ist eine Penalisierung von größeren Abweichungen gewünscht, um so schnell an das Optimum navigieren. Dies ermöglicht der Mean Squared Error (MSE), dieser entspricht der RSS dividiert durch die Anzahl der Beobachtungen und ist somit der durchschnittliche quadrierte Fehler:
\begin{equation}
  \text{MSE} = \frac{1}{n} \sum_{i=1}^{n} (y_i - \hat{y}_i)^2 = \frac{RSS}{n}
  \label{eq:mse}
\end{equation}
Hierbei \(y_i\) die beobachteten Werte, \(\hat{y}_i\) die vorhergesagten Werte und \(n\) die Anzahl der Beobachtungen darstellt.

\subsubsection{Root Mean Squared Error (RMSE)}
Der MSE lässt sich aufgrund der quadratischen Form nicht intuitiv interpretieren, dies lösst der Root Mean Squared Error (RMSE). Der RMSE ist die Quadratwurzel des MSE und kann so in der selben Einheit wie die Zielvariable interpretiert werden:

\begin{equation}
  \text{RMSE} = \sqrt{\text{MSE}} = \sqrt{\frac{1}{n} \sum_{i=1}^{n} (y_i - \hat{y}_i)^2}
  \label{eq:rmse}
\end{equation}

Diese Eigenschaft macht den RMSE wertvoll für die praktische Interpretation der Modellgüte, da die durchschnittliche Abweichungen zwischen Vorhersage und den tatsächlichen Werten direkt angegeben wird. 

\subsubsection{Mean Absolute Error}
Die Quadratisierung der Werte für zu einem besondren Fokuss auf die großen, stark abweichenden Werte, diese eigenet sich für die mathematische Optimierung und für Datensätze ohne extreme Werte. Für Datensätze mit solchen extremen Werten eignet sich der Mean Absolute Error (MAE). Dieser verwendet die absoluten Abweichungen und ist so weniger sensitiv auf Ausreißer:

\begin{equation}
  \text{MAE} = \frac{1}{n} \sum_{i=1}^{n} |y_i - \hat{y}_i|
  \label{eq:mae}
\end{equation}

\subsubsection{Mean Absolute Precentage Error (MAPE)}

Fehlermaße die eine Einheit tragen, lassen sich schlecht untereinander zwischen verschiedenen Datensätzen vergleichen, da die Zielvariablen im Allgemeinen in verschiedenen Größen vorliegen werden. Der Mean Absolute Precentage Error (MAPE) drückt den durchschnittliche absoluten Fehler als Prozentsatz aus und ermöglicht dadurch eine einheitenlose Bewertung der Modellgüte.

\begin{equation}
\text{MAPE} = \frac{1}{n} \sum_{i=1}^{n} \left|\frac{y_i - \hat{y}_i}{y_i}\right| \times 100\%
\label{eq:mape}
\end{equation}

\subsection{Lineareregression}
Die Lineareregression ist ein Modell aus der Statistik welches den Zusammenhang zwischen einer Antwortvariable \(Y\) und einer erklärenden Varaible \(X\) herstellt. Das verwendete Modell ist ein lineares Modell es wird also die Annahme getroffen, dass zwischen \(Y\) und \(X\) ein linearer Zusammenhang vorliegt. Für eine einzelne erklärende Variable \(X\) ergeben sich zwei unbekannte Konstanten \(\beta_0\) und \(\beta_1\), diese zubestimmen ist das Ziel der linearen Regression. Das Modell lässt sich somit schreiben als:
\begin{equation}
    Y \simeq \beta_0 + \beta_1 X
    \label{eq:linreg1}
\end{equation}
Es ist zu unterscheiden zwischen den Modell Koeffizienten \(\beta\) und den Schätzungen dieser Koeffizienten \(\hat{\beta}\), die sich aus den verwendeten Trainingsdaten ergeben. 
Für das lineare Modell lässt sich ein direkter Ausdruck für die geschätzen Werte bilden, ausgedrückt über den Mittelwert von \(Y\) und \(X\):
\begin{equation}
    \hat{\beta_1} = \frac{\sum_{i=1}^{n}(x_i - \overline{x})(y_i - \overline{y})}{\sum_{i=1}^{n}(x_i - \overline{x})^2}
\end{equation}
\begin{equation}
    \hat{\beta_0} = \overline{y} - \hat{\beta_1}\overline{x}
\end{equation}

\cite{james2013}

\subsection{Refularisierte Lineareregression}

\subsubsection{Ridge Regression (L2-Regularisierung)}
Wenn in dem Vektor der erklärenden Variablen X viele miteinander korrelierte Werte vorliegen, dann kann die L2-Regularisierung verwendet werden, diese führt dazu dass korrelierte Variablen ähnliche Koeffizienten erhalten. Diese Methode ist Ridge Regression, also die Kombination aus Lineareregression und einem weiteren L2-Term. Die Kostenfuktion, hier der RSS, wird um einen weitren Strafterm erweitert.

\begin{equation}
\text{RSS}_{\text{Ridge}} = \sum_{i=1}^{n}(y_i - \beta_0 - \sum_{j=1}^{p}\beta_j x_{ij})^2 + \lambda \sum_{j=1}^{p}\beta_j^2
\label{eq:rss-ridge}
\end{equation}
Der Hyperparameter \(\lambda \geq 0\) kontrolliert die Stärke der L2-Regularisierung. Bei \(\lambda = 0\) entspricht die Ridge-Regression der Lineareregression. Größere Werte von \(\lambda \) reduzieren die Größe der Koeffizienten.

\subsubsection{Lasso Regression}
Die Lineareregression eignet sich nicht für hoch dimensionale Datensätze. Wenn viele dieser weiteren Werte irrelevant für die Regression sind, dann kann eine L1-Regularisierung verwendet werden, da diese Koeffizienten auf null setzen kann. Diese Variante der Lineareregression ist die Lasso Regression (Least Absolute Shrinkage and Selection Operator). Sowohl die L1 als auch die L2 Norm, führen neben diesen Eigenschaften zu einer Regularisierung und reduzieren beide so overfitting.

\begin{equation}
\text{RSS}_{\text{Lasso}} = \sum_{i=1}^{n}(y_i - \beta_0 - \sum_{j=1}^{p}\beta_j x_{ij})^2 + \lambda \sum_{j=1}^{p}|\beta_j|
\label{eq:reg-lasso}
\end{equation}

\subsubsection{Elastic Net}
Die Vorteile der Ridge und der Lasso Regression lassen sich kombinieren, durch die Verwendung beider Regressionsterme.

\begin{equation}
\text{RSS}_{\text{Elastic Net}} = \sum_{i=1}^{n}(y_i - \beta_0 - \sum_{j=1}^{p}\beta_j x_{ij})^2 + \lambda_1 \sum_{j=1}^{p}|\beta_j| + \lambda_2 \sum_{j=1}^{p}\beta_j^2
\label{eq:reg-elastic}
\end{equation}

\subsection{Baumbasierte Verfahren}

\subsubsection{Entscheidungsbäume}

Wenn nicht lineare Zusammenhänge vorliegen, eignen sich lineare Modelle nicht mehr. Entscheidungsbäume sind ein Verfahren, dass für nicht lineare Zusammenhänge verwendet werden, kann und trozdem eine hervorangende Interpretierbarkeit beibehält. Ein Entscheidungsbäum teilt den Merkmalsraum rekursiv in rechteckige Regionen auf, jede dieser Regionen wird durch eine einfach Regel, die nur auf einem Merkmal basiert, charakterisiert. Die Antwortvariable wird durch die Mittelwertbildung der Endregion \(R_j\) bestimmt:

\begin{equation}
\hat{y}_{R_j} = \frac{1}{|R_j|} \sum_{x_i \in R_j} y_i
\label{eq:reg-tree}
\end{equation}

Die Optimierung erfolgt, durch die Minimierung der Summe der quadratischen Abweichungen in den Teilregionen bestimmt. Die gute Interpretierbarkeit muss mit der hohen Varianz und der Neigung zum Overfitting abgewogen werden.

\subsubsection{Random Forest}
Um das Overfitting und die Varianz eines einzelnen Baumes zu verbessern, wird der Random Forest verwendent, dieser ist ein Ensemble-Verfahren, das die finale Vorhersage von mehren einzel Bäumen kombiniert \ref{eq:reg-random}. Es werden zwei Randomisierungsstrategien verwendet, zum einem eine bootstrap Aggregation (Bagging), also jeder Baum wird auf einer Bootstrap-Stickprobe der Trainingsdaten trainiert und zum anderen die Feature Randomisierung, es wird in jeder Aufteilung des Merkmalsraums nur eine Teilmenge von den Variablen betrachtet. 

\begin{equation}
  \hat{y}_{\text{RF}} = \frac{1}{B} \sum_{b=1}^{B} \hat{y}_b
  \label{eq:reg-random}
\end{equation}

Der Random Forest eignet sich als gute Wahl für viele Regressionsaufgaben, die nicht linear sind, jedoch ist dieser weniger Interpretierbar als, ein einzelner Entscheidungsbaum.

\subsection{Gradient Boosting}
Neben dem Random Forest, also der Wahl von mehren tiefen Bäumen, lassen sich auch schwache Lerner, z.B. flache Entscheidungsbäume verwenden. Dabei baut das Gradient Boosting Model sequenziell solche schwachen Lerner auf, wobei neue weitere Bäume die Fehler der vorheringen Iteration korrigieren. Es wird iterativ eine Verlustfunktion minimiert:
\begin{equation}
  F_m(x) = F_{m-1}(x) + \gamma_m h_m(x)
  \label{eq:reg-grad}
\end{equation}

Gradient Boosting führt häufig zu hervorragenden Modellen, jedoch muss auf eine sorgfältige Hzperparameter-Abstimmung geachtet werden, um Overfitting zu Vermeiden. 


% Clustering Verfahren kmeans, hdbscan, pca
% Features (statisticle, fft)
% deeplearning (cnn, lstm, pinn)
% Dimensionsreduktioin (PCA, UMAP, t-SNE)
