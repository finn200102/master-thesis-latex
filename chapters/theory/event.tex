\subsection{Ereignis - Zustandsänderung}
\label{sec:theory-event}

Ein Ereignis in einem thermodynamischen System ist eine zeitliche und räumliche lokalisierte Zustandsänderung die durch spezifische physikalische Prozesse ausgelöst wird. Es lassen sich somit 3 zeitliche Regionen definieren, der Zustand vor dem Ereignis, das Ereignis und der Zustand nach dem Ereignis. Für eine solche lokalisierte Zustandsänderung stimmt der erste Zustand mit dem zweiten überein, für Phasenübergänge würde sich der erste von dem zweiten Zustand unterscheiden, da das vollständige System verändert wurde.

\subsection{Thermodynamisches Gleichgewicht}

\subsection{Zustandsänderung}

\subsection{Ereignisdynamik}
