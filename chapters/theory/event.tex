\subsection{Ereignis - Zustandsänderung}
\label{sec:theory-event}

Ein Ereignis in einem thermodynamischen System ist eine zeitliche und räumliche lokalisierte Zustandsänderung die durch spezifische physikalische Prozesse ausgelöst wird. Es lassen sich somit 3 zeitliche Regionen definieren, der Zustand vor dem Ereignis, das Ereignis und der Zustand nach dem Ereignis. Für eine solche lokalisierte Zustandsänderung stimmt der erste Zustand mit dem zweiten überein, für Phasenübergänge würde sich der erste von dem zweiten Zustand unterscheiden, da das vollständige System verändert wurde.

\subsection{Thermodynamisches Gleichgewicht}
Ein System liegt in einem thermodynamischen Gleichgewicht vor, wenn es keine makroskopische Nettoänderung in Größen wie des Drucks, der Temperatur etc. mehr gibt. Ein solches System bleibt in diesem konstanten Zustand solange bestehen bis eine thermodynamische Operation von Außen auf das System angewendt wird z.B. durch das Heizen einer Seite einer Box, die hier das System darstellt, das so entstehende Potential (ein Temperaturgradient) stellt eine Auslenkung von dem vorheringen thermodynamischen Gleichgewicht da, da alle Systeme ein thermodynamisches Gleichgewicht anstreben nach dem zweiten Haupsatz der Thermodynamik, treten nun makroskopische Austauschprozesse auf, die diese Potentiale wieder minimieren und so erneut ein Gleichgewicht bilden. Wie mit dem Beispiel der Box wird ersichtlich, dass beachtet werden muss, von welchem Subsystem gesprochen wird. Es kann lokal ein thermodynamisches Gleichgewicht vorliegen ohne dass das ganze System sich in einem solchen befindet.

\subsection{Zustandsänderung}
Ein System kann sich in einem thermodynamischen Gleichgewicht befinden in dem keine makroskopische Austauschprozesse stattfinden, jedoch kann ein Punkt erreicht werden an dem durch eine kleine Änderung z.B. des chemischen Potentials eine Reaktion stattfinden kann bei der genügend Energie frei wird, dass das Gleichgewicht zumindest lokal gestört wird. Wenn die darauffolgende Thermalisierung das System genügend verändert z.B. Übergang zwischen zwei Aggregatzuständen, dann ist das System in einen anderen Zustand übergegangen. Wenn jedoch nach einer solchen Störung das System wieder in den annähernden Anfangszustand übergeht, da die Störung zu klein war bzw. das lokale System im Austausch mit einem sehr viel größeren System steht z.B. Wärmebad, bei solchen Phänomenen handelt es sich um transientes Phänomen. In dieser Arbeit werden solche transienten Erscheinung als Events bezeichnet. Hierbei ist der hier verwendete Event begriff nicht mit dem aus der Teilchenphysik oder der Relativitätstheorie zu verwechseln.

\subsection{Ereignisdynamik}

\subsection{Kritikalität - Kippunkte}
