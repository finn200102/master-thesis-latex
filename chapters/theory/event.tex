\section{Ereignis - Zustandsänderung}
\label{sec:theory-event}

Ein Ereignis in einem thermodynamischen System ist eine zeitliche und räumliche lokalisierte Zustandsänderung, die durch spezifische physikalische Prozesse ausgelöst wird. Es lassen sich somit drei zeitliche Regionen definieren, der Zustand vor dem Ereignis, das Ereignis selbst und der Zustand nach dem Ereignis. Bei einer lokalisierten transienten Störung kehrt das System in seinen Ausgangszustand zurück, während bei einer permanenten Zustandsänderung, wie etwa einem Phasenübergang, das System in einen qualitativ neuen Zustand übergeht.

\subsection{Thermodynamisches Gleichgewicht}
Ein System liegt in einem thermodynamischen Gleichgewicht vor, wenn es keine makroskopische Nettoänderung in Größen wie Druck, Temperatur oder Teilchendichte mehr auftreten. Ein solches System bleibt in diesem konstanten Zustand solange bestehen bis eine thermodynamische Operation von Außen auf das System angewendt wird z.B. durch das Heizen einer Seite einer Box, die hier das System darstellt, das so entstehende Potential (ein Temperaturgradient) stellt eine Auslenkung von dem vorheringen thermodynamischen Gleichgewicht da, da alle Systeme ein thermodynamisches Gleichgewicht anstreben nach dem zweiten Haupsatz der Thermodynamik, treten nun makroskopische Austauschprozesse auf, die diese Potentiale wieder minimieren und so erneut ein Gleichgewicht bilden. 
Wichtig ist die Unterscheidung zwischen lokalem und globalen Gleichgewicht. Ein System kann sich global im Nichtgleichgewicht befinden, während lokal quasistationäre Zustände existieren. Diese lokalen Gleichgewichte ermöglichen die Beschreibung von Ereignissen als Störungen eines metastabilen Zustands. In Plasmen können solche lokalen thermodynamischen Gleichgewichte existieren.

\subsection{Zustandsänderung}
Die Stabilität eines thermodynamischen Gleichgewichts wird durch die Potentiallandschaft des Systems bestimmt. Kleine Fluktuationen um einen stabilen Zustand klingen exponentiell ab, ein Überschreiten einer kritischen Schwelle führt jedoch zu einer qualitativen Änderung des Systems. Diese Schwelle kann durch verschiedene Parameter definiert werden, wie dem chemischen Potential, dem elektrischen Feld, der Temperatur oder dem Druck.

Transiente Ereignisse sind charakterisiert durch eine temporäre Auslenkung aus dem Gleichgewicht mit anschließender Relaxation in den Ausgangszustand. Die Zeitskala dieser Relaxation wird durch die auftretenden dominanten Dissipationsmechanismen bestimmt. Permanente Zustandsänderungen führen im Gegensatz zu einem neuen stabilen Gleichgewicht. Das Verhältnis aus der freigesetzten Energie und der Dissipationskapazität der Umgebung charakterisiert die Unterscheidung dieser beiden Prozesse. Im Kontext von elektrischen Entladungen tritt ein solcher Unterschied zwischen einer sich selbst löschenden Teilentladung und einem sich selbst erhaltenden Durchbruch auf.

\subsection{Ereignisdynamik}
Die zeitliche Evolution eines Ereignisses lässt sich in drei charakteristische Phasen unterteilen, die Initiierung, die Propagation und die Relaxation. Inter der Initiierungsphase überschreitet eine lokale Fluktuation des Systems eine kritische Schwelle, dies führt zu einer postivien Rückkopplung. Die Propagationsphase zeichnet sich durch eine nichtlineare Dynamik aus, die zu der Verstärkung und Ausbreitung der anfänglich kleinen Störung führt.

In der räumlich-zeitlichen Ausbreitung konkurieren die treibenden und dämpfenden Mechanismen. Die Ausbreitungsgeschwindigkeit hängt von den relevanten Transportkoeffizienten ab. Die Diffusion führt zu einer \(\sqrt{t}\) Skalierung der Ausbreitung, während konvektive oder wellenartige Ausbreitungen zu einer linearen Zeitabhängigkeit führen. Im Kontext von ionisierten Gasen wirken sowohl Diffusionsprozesse als auch Driftbewegunen im elektrischen Feld. 

Die Relaxationsphase beginnt, wenn die treibende Kraft erschöpft ist oder durch negative Rückkopplungsmechanismen kompensiert wird. Die charakterische Relaxationsyeit \(\tau\) bestimmt, ob aufeinanderfolgende Ereignisse als unabhängig betrachtet werden können oder ob das letzte Ereignis das jetzige beinflust.

\subsection{Kritikalität - Kippunkte}
Kritische Punkte sind besondere Zustände, an denen ein System zwischen zwei grunlegend verschiedenen Verhaltensweisen wechseln kann. An diesen Punkten reagieren Systeme extrem empfindlich auf kleinste Störungen. Diese erhöhte Empfindlichkeit des Systems zeigt sch mathematisch durch divergierende Korrelationslängen und Potenzgesetze, die universell in verschiedensten Systemen auftreten. 

Ein Phänomen ist die selbstorganisierte Kritikalität. Hierbei entwickelt sich eine System selbst in einen kritischen Zustand, ohne das äußere Parameter präzise eingestellt werden müssen. Im Kontext der Elektrischenentladungen folgt eine Elektronenlawine dieser Struktur. Ein Starter Elektron initiert einen solchen Lawinenprozess, der einem Potenzgesetz der Form \(P(s) \sim s^{-\alpha}\) folgt, hiert ist s die Anzahl der erzeugten Ladungsträger.

Kippunkte sind eine spezielle Art kritischer Übergänge. Sie führen zu abrupten Systemänderungen. Das System befindet sich in einem scheinbar stabilen Zustand, bis eine kritische Schwelle überschritten wird und sich eine Kettenreaktion bildet, wie auch bei einer elektrischen Entladung.

Solche Kippunkte kündigen sich durch charakteristische Warnsignale an. Das System braucht länger um nach Störungen ins Gleichgewicht zurückzukehren (critical slowing down), die Schwankungen werden größer und bleiben länger bestehen. Diese Signale ermöglichen prinzipiell eine Vorhersage bevorstehender Übergänge. 
