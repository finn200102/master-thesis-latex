\subsection{Rekombinationsmechanismen}
\subsubsection{Elektron-Ion-Rekombination}
Wenn Elektronen und positiv geladene Ionen in Form eines Stoßes interagieren, kann es zu einer Rekombination kommen, d. h., es bildet sich wieder ein neutrales Teilchen. Die so frei werdende Energie, die der kinetischen Energie des Elektrons entspricht, muss diesem System entweichen; dies geschieht über verschiedene Prozesse.\newline\newline
\textbf{Strahlungsrekombination}
In der Strahlungsrekombination stößt ein Elektron mit einem Ion und diese rekombinieren; die übrige Energie wird in Form eines Photons abgestrahlt. Dieser Prozess ist noch nicht notwendigerweise zu Ende, denn das so entstandene neutrale Atom kann sich noch in einem angeregten Zustand befinden. Die Abregung dieses Zustands führt dann ebenfalls zur Abstrahlung eines Photons:
\begin{equation}
    A^+ + e^- \rightarrow A^* + hf \rightarrow A + hf' + hf
    \label{eq:radiationrecombination}
\end{equation}\newline
\textbf{Dissoziative Rekombination}
In diesem Prozess stößt ein molekulares Ion mit einem Elektron, und die frei werdende Energie in der Rekombination spaltet das Molekül in zwei Teilchen. Der Prozess der Spaltung ist nicht instantan, sondern es entsteht zuerst ein angeregtes Molekül, das sich anschließend durch die Spaltung abregt und die Energie in Form von kinetischer Energie auf die Spaltprodukte überträgt. In molekularen Gasen, in denen dieser Prozess möglich ist, kommt es so zu einer sehr viel schnelleren Rekombination als in atomaren Gasen (ca. zwei Größenordnungen), da die Übertragung der kinetischen Energie des Elektrons auf angeregte Vibrationszustände sehr schnell erfolgt.
\begin{equation}
    AB^+ + e^- \leftrightarrow AB^* \leftrightarrow A^* + B + E_{kin}
\end{equation}\newline
\textbf{Dreikörperrekombination}
In der Dreikörperrekombination wird die übrige Energie an einen dritten Körper \(M\) weitergegeben, also an ein weiteres Teilchen. Bei niedrigen Drücken ist die Wahrscheinlichkeit, einen dritten Körper im Volumen zu finden, sehr klein, sodass dieser Prozess fast ausschließlich an den Wänden des Volumens stattfindet, die also die kinetische Energie aufnehmen.
\begin{equation}
    A^+ + e^- + M \rightarrow A + M
\end{equation}

