\subsection{Rekombinationsmechanismen}
\subsubsection{Elektron-Ion-Rekombination}
Wenn Elektronen und positiv geladene Ionen in Form eines Stoßes interagieren, kann es zu einer Rekombination kommen, d.h. es bildet sich wieder ein neutrales Teilchen. Die so frei werdene Energie die der kinetischen Energie des Elektrons entspricht muss diesem System entweichen, dies geschieht über verschiedene Prozesse.\newline\newline
\textbf{Strahlungsrekombination}
In der Strahlungsrekombination stößt ein Elektron mit einem Ion und diese rekombinieren, die übrige Energie wird in Form eines Photons abgestrahlt, dieser Prozess ist noch nicht notwendigerweise zu Ende, denn das so entstandene neutrale Atom kann sich noch in einem angeregten Zustand befinden. Die Abregung dieses Zustandes führt dann ebenfalls zur Abstrahlung eines Photons:
\begin{equation}
    A^+ + e^- \rightarrow A^* + hf \rightarrow A + hf' + hf
    \label{eq:radiationrecombination}
\end{equation}\newline
\textbf{Dissoziative Rekombination}
In diesem Prozess stößt ein molekulares Ion mit einem Elektron und die frei werdene Energie in der Rekombination spaltet das Molekühl in zwei Teilchen. Der Prozess der Spaltung ist nicht instantan, sonder es entsteht zuerst ein angeregtes Molekühl das sich anschießend durch die Spaltung abregt und die Energie auf in Form von kinetischer auf die Spaltprodukte überträgt. In molekularen Gasen, indenen dieser Prozess möglich ist, kommt es so zu einer sehr viel schnelleren Rekombination als in atomaren Gasen (ca. 2 Größenordungen), da die Übertragung der kinetischen Energie des Elektrons auf angeregte Vibrationszustände sehr schnell geht.
\begin{equation}
    AB^+ + e^- \leftrightarrow AB^* \leftrightarrow A^* + B + E_{kin}
\end{equation}\newline
\textbf{Dreikörper Rekombination}
In der dreikörper Rekombination wird die übrige Energie an einen dritten Körper \(M\) weiter gegeben, also einem weiteren Teilchen. Bei niedrigen Drücken ist die Wahrscheinlichtkeit einen dritten Körper im Volumen zu finden sehr klein, so dass dieser Prozess fast auschließlich an den Wänden des Volumens stattfindet, also an diese die kinetische Energie abgeben.
\begin{equation}
    A^+ + e^- + M \rightarrow A + M
\end{equation}

