\subsection{Kathodenprozesse}
Entladungen unter experimental Bedingungen im Labor finden zwischen zwei Elektroden statt. Die notwendigen Elekronen für eine Entladung können über verschiedene Mechanismen aus der Elektrode gelöst werden, die dabei notwendige Energie ist durch die Arbeitsfunktion des Elektrodenmetalls gegeben:
\begin{equation}
    W = -e \phi - E_f
\end{equation}
Hierbei ist \(\phi\) das elektrostatische Potential und \(E_f\) das Fermienergielevel. Diese Barriere muss überkommen werden durch das Einwirken von Energie auf die Elektrode um die Elektronen direkt herauszulösen oder durch das herabsenken der Barriere durch das anlegen eines elektrischen Feldes.
\subsubsection{Photoelektrische Emission}
Die photoelektrische Emission, erstmals von Heinrich Hertz 1887 beobachtet und später von Albert Einstein theoretisch erklärt, beschreibt den Prozess, bei dem Elektronen durch Lichteinwirkung aus einer Metalloberfläche herausgelöst werden.
\paragraph{Physikalischer Mechanismus}
Ein einfallendes Photon mit der Frequenz \(f\) überträgt seine gesamte Energie \(E_p = hf\) auf ein einzelnes Elektron im Metall. Dabei wird die Energie vollständig absorbiert. Es handelt sich um einen quantenmechanischen Prozess ohne klassische Entsprechung. Das Elektron erhält diese Energie instantan und kann, sofern sie ausreicht, die Potentialbarriere der Arbeitsfunktion \(W\) überwinden. Die kinetische Energie des austretenden Elektrons ergibt sich aus der Energieerhaltung: 
\begin{equation}
    E_e = hf - W
\end{equation}
\paragraph{Eigenschaften}
Beim photoelektrischen Effekt gilt Folgendes: Unterhalb der materialspezifischen Schwellenfrequenz \(f_0\) tritt keine Elektronenemission auf, und das unabhängig von der Intensität des einfallenden Lichts. Die maximale kinetische Energie der emittierten Photoelektronen hängt ausschließlich von der Frequenz des Lichts ab und nicht von seiner Intensität. Dagegen ist die Anzahl der emittierten Elektronen, also der Photostrom, direkt proportional zur Lichtintensität. Schließlich erfolgt die Elektronenemission praktisch verzögerungsfrei, sobald das Material beleuchtet wird. \cite{cooray2014}
\subsubsection{Thermische Emmision}
Das heizen der Elektrode erhöht die thermische und somit kinetische Energie der Elektroden in dem Metall und senkt somit die notwendige Energie das Material zu verlassen. Wenn die Temperatur hoch genug ist, können die Elektronen die Potentialbarriere allein überwinden und das Metall verlassen.
\subsubsection{Schottky Effekt}
Die Barriere die die Elektronen überwinden müssen lässt sich durch das anlegen eines richtig ausgerichtetem elektrischen Feldes verringern. Dies lässt sich mit der thermischen Emission kombinieren. Die Elektronen verspüren nun eine Kraft die aus dem Metall herausgerichtet ist. Dieser Effekt wird z.B. in Röhrenfernsehern verwendet um einen Strahl aus Elektronen mit hoher kinetischer Energie zu realisieren. \cite{cooray2014}
