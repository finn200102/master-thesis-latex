\subsection{Ionisationsmechanismen}
Ein Gasteilchen kann durch verschiedene Mechanismen ionisiert werden. Bei all diesen Mechanismen interagiert das zu ionisierende Teilchen mit mindestens einem weiteren Teilchen, das die notwendige Energie zur Verfügung stellt, um das Elektron aus der Hülle des Atoms zu heben. Die dazu notwendige Energie ist die Ionisationsenergie \(E_i\), die als Differenz der Energie des ionisierten und des neutralen Atoms definiert ist \eqref{eq:energy-ion}. Für Wasserstoff beträgt die Ionisationsenergie \(\SI{13,6}{\electronvolt}\), entsprechend der Rydberg-Konstante.

\begin{equation}
    E_{i} = E^{*} - E
    \label{eq:energy-ion}
\end{equation}

\subsubsection{Elektron-Stoßionisation}
Ein durch ein elektrisches Feld beschleunigtes Elektron kann in einem unelastischen Stoß ein Quantum seiner Energie auf ein Atom übertragen. Wenn diese Energie größer als die Ionisationsenergie ist, kann das Elektron das Atom ionisieren. Dieser Prozess kann auch stufenweise geschehen, d. h., das erste Elektron regt das Atom an und hebt ein Elektron in einen Zustand höherer Energie, und das zweite stellt nun die restliche Energie zur Verfügung, um das Atom zu ionisieren. Dieser Prozess der Ionisation lässt sich über den Ionisations-Wirkungsquerschnitt \(\sigma_{ion}\) der Atome und die Dichte \(n\) der Atome quantifizieren. Das Produkt aus diesen beiden Größen heißt Koeffizient der Entladung bzw. erster Townsend-Koeffizient; dieser ist in \eqref{eq:townsedfirst} dargestellt.
\begin{equation}
    \alpha = \sigma_{ion} n
    \label{eq:townsedfirst}
\end{equation}
Der inverse Wert des Koeffizienten der Entladung ist die mittlere freie Weglänge \(\lambda_{ion}\), die Strecke, die ein Elektron zurücklegen kann, bevor es ein Atom ionisiert. In \eqref{eq:middelwaylen} ist diese Größe dargestellt.

\begin{equation}
    \lambda_{ion} = \frac{1}{\alpha} = \frac{1}{\sigma_{ion} n}
    \label{eq:middelwaylen}
\end{equation}
Die Wahrscheinlichkeit, dass es zur Ionisation kommt, lässt sich bestimmen, wenn der totale Wirkungsquerschnitt \(\sigma\) bekannt ist.
\begin{equation}
    P_{ion} = \frac{\sigma_{ion}}{\sigma}
    \label{eq:ionizationprop}
\end{equation}
Somit entspricht die Ionisationswahrscheinlichkeit dem Quotienten aus dem Ionisationsquerschnitt und dem totalen Wirkungsquerschnitt. Die Ionisationskonstante lässt sich über die folgende Formel bestimmen:
\begin{equation}
    \frac{\alpha}{p} = A e^{-\frac{B}{\frac{E}{p}}}
    \label{eq:ionconst}
\end{equation}
Hierbei ist \(p\) der Druck, \(E\) das elektrische Feld und \(A\) und \(B\) Konstanten, die von dem verwendeten Gas abhängen \cite{cooray2014}.

\subsubsection{Photoionisation}
Auch Photonen können die notwendige Energie auf ein Atom übertragen, um dieses in einen Zustand höherer Energie zu überführen. Wenn diese Energie \(E_{p} = \mathrm{h}f\) größer als die Ionisationsenergie des Atoms \(E_{ion}\) ist, kommt es zur Ionisation. Hierbei ist \(\mathrm{h}\) die plancksche Konstante und \(f\) die Frequenz des Photons. Die kinetische Energie des Elektrons ist somit:
\begin{equation}
    E_{e} = \mathrm{h}f - E_{ion}
    \label{eq:photoenergyion}
\end{equation}
Wenn die Energie geringer als die Ionisationsenergie ist, kann das Atom jedoch angeregt werden und ein Elektron in einen Zustand höherer Energie versetzt werden, sodass ein weiteres Photon die nun verringerte Ionisationsenergie aufbringen kann, um das Atom zu ionisieren. Dies ist die Mehrphotonionisation, wobei die Wahrscheinlichkeit etwa mit \(I^n\) skaliert, wenn \(n\) Photonen absorbiert werden \cite{kuffel2000}.

\subsubsection{Thermische Ionisation}
\label{sec:thermion}
Ein Gas mit steigender Temperatur entspricht auch einer steigenden kinetischen Energie der Teilchen. Die Geschwindigkeit der Atome in einem Gas wird durch die Maxwell-Boltzmann-Verteilung beschrieben; sie folgt also einer statistischen Verteilung. Somit gibt es Ausreißer in diesem System, die schon früher eine so hohe kinetische Energie besitzen, dass es zu Kollisionen mit genügender Energie kommt, um Atome zu ionisieren. Sobald es zu Ionisationen kommt, gibt es mehrere Prozesse, die dafür verantwortlich sind: die Kollision von neutralen Atomen mit anderen neutralen Atomen, die Kollisionen von thermischen Ionen mit neutralen Atomen, die Kollision von Elektronen mit neutralen Atomen. Die Elektronen wurden hierbei durch die oberen Ionisationsmechanismen erzeugt, also Elektron-Stoßionisation. Mit der Annahme, dass alle diese Teilchen und Prozesse im thermischen Gleichgewicht miteinander stehen, lässt sich die Boltzmann-Verteilung auf die verschiedenen Teilchen einzeln anwenden. Damit lässt sich die Saha-Gleichung formulieren:
\begin{equation}
    \frac{\beta^2}{1 - \beta^2} = \frac{2,4 \cdot 10^{-4}}{p}T^{2,5}e^{\frac{-E_{ion}}{kT}}
    \label{eq:saha}
\end{equation}
Hierbei ist \(\beta = \frac{n_i}{n}\) das Verhältnis aus der Anzahl ionisierter Teilchen \(n_i\) und der totalen Anzahl an Teilchen. Die Saha-Gleichung zeigt, dass die thermische Ionisation in Luft erst ab ca. \(\SI{4000}{\kelvin}\) eine Rolle spielt \cite{kuffel2000}.

\subsubsection{Ionstoßionisation}
Auch geladene Teilchen neben Elektronen, also Ionen, werden in einem elektrischen Feld beschleunigt und gewinnen so kinetische Energie. Ein Stoß zwischen einem Ion mit genügender kinetischer Energie und einem neutralen Atom kann ebenfalls zu einer Ionisation führen. Es wird jedoch im Vergleich zum Elektron eine höhere Energie benötigt, da der Energieübertrag wegen der größeren Masse weniger effizient ist, im Gegensatz zum Elektronstoß, bei dem nur etwa die Ionisationsenergie benötigt wird. Dies führt zu einer mehr als doppelt so hohen benötigten Energie des Ions.

