\section{Verteilung der Eventeigenschaften}
\label{sec:event_value_distribution}

\subsection{Event-Ladung und -Energie}
\label{sec:event_charge_energie}
Die Verteilung der in \secref{sec:eventenergy_charge} definierten Eventenergie und -ladung für die beiden Elektrodenformen ist in den Abbildungen \ref{fig:box-gap-eventcharge-stab} bis \ref{fig:box-gap-event-charge-energy} dargestellt. Wie nach der Definition dieser beiden Größen zu erwarten, stimmen die Verteilungen weitgehend überein. Es sind keine klaren Trends weder für die Stabelektrode noch für die Plattenelektrode erkennbar. Zwar zeigen die Mittelwerte pro Spaltabstand schwache Tendenzen, jedoch wird der in \figref{fig:box-gap-event-charge-plate} von \SI{1e-8}{\coulomb} nach \SI{1e-9}{\coulomb} verlaufende Trend durch starke Schwankungen der dazwischenliegenden Spaltabstände überlagert; Gleiches gilt für die Eventenergie in \figref{fig:box-gap-event-energy-plate}. Für die Stabelektroden ist für beide Größen ebenfalls eine leichte Tendenz erkennbar, die jedoch von kleineren Energien um \SI{1e-10}{\joule} zu größeren Energien um \SI{1e-9}{\joule} verläuft, ohne dass ein kontinuierlicher Trend vorliegt. Die stärkste Änderung zeigt sich in beiden Fällen zwischen den ersten beiden Spaltabständen \SI{0.1}{\centi\meter} und \SI{0.3}{\centi\meter} und den übrigen für die Stabelektrode sowie zwischen dem ersten Spaltabstand \SI{0.3}{\centi\meter} und den übrigen für die Plattenelektrode. Der Unterschied zwischen den Werten für den Spaltabstand von \SI{0.1}{\centi\meter} und den übrigen lässt sich teilweise dadurch erklären, dass hierfür die größte Statistik vorliegt. Die stärkere Änderung zwischen den ersten beiden Spaltabständen deutet jedoch darauf hin, dass für größere Abstände ein gewisses Plateau erreicht wird. Für die Stabelektroden mit der größeren Statistik ist dies der erwartete Verlauf: Mit zunehmendem Spaltabstand nehmen Energie bzw. Ladung im Mittel zu, zumindest bis dieses Plateau erreicht ist. Für die Plattenelektroden zeigt sich hingegen der umgekehrte Verlauf, das heißt, Energie bzw. Ladung fallen auf das Plateau ab und werden somit kleiner. Das Plateau liegt für die Eventenergie bei beiden Elektrodenformen bei etwa \SI{1e-9}{\joule}. Die Ladung liegt hingegen für alle Spaltabstände bis auf \SI{1}{\centi\meter} bei der Stabelektrode unter \SI{1e-10}{\coulomb}; für die Plattenelektrode liegen die Mittelwerte bis auf \SI{2.5}{\centi\meter} unter dieser Schwelle.

\begin{figure}[H]
  \centering
  \begin{subfigure}[t]{0.48\textwidth}
    \centering
    \includegraphics[width=\textwidth]{figures/classes/event_charge_box_electrodes_gap.pdf}
    \caption{Die Ladung des Events aufgetragen gegen den Spaltabstand d für die Stabelektrode}
    \label{fig:box-gap-eventcharge-stab}
  \end{subfigure}
  \begin{subfigure}[t]{0.48\textwidth}
    \centering
    \includegraphics[width=\textwidth]{figures/classes/event_charge_box_plates_gap.pdf}
    \caption{Die Ladung des Events aufgetragen gegen den Spaltabstand d für die Plattenektrode}
    \label{fig:box-gap-event-charge-plate}
  \end{subfigure}
  \centering
  \begin{subfigure}[t]{0.48\textwidth}
    \centering
    \includegraphics[width=\textwidth]{figures/classes/event_energy_box_electrodes_gap.pdf}
    \caption{Die Energie des Events aufgetragen gegen den Spaltabstand d für die Stabelektrode}
    \label{fig:box-gap-event-energy-stab}
  \end{subfigure}
  \begin{subfigure}[t]{0.48\textwidth}
    \centering
    \includegraphics[width=\textwidth]{figures/classes/event_energy_box_plates_gap.pdf}
    \caption{Die Energie des Events aufgetragen gegen den Spaltabstand d für die Plattenektrode}
    \label{fig:box-gap-event-energy-plate}
  \end{subfigure}
  \caption{Verteilung der Eventladung und der Energie}
  \label{fig:box-gap-event-charge-energy}
\end{figure}

\subsection{Event-Durchbruchspannung und-Spannungsdifferenz}
In \figref{fig:box-gap-event-voltage-stab} und \figref{fig:box-gap-event-voltage-plate} sind die Verteilungen der Durchbruchspannung \(U_B\) in Abhängigkeit vom Spaltabstand dargestellt. Wie zu erwarten, ist ein Trend von kleineren Durchbruchspannungen von \SI{1}{\kilo\volt} bis über \SI{10}{\kilo\volt} erkennbar, mit Ausnahme der Spaltabstände \SI{6}{\milli\meter}, \SI{10}{\milli\meter} und \SI{11}{\milli\meter} für die Stabelektrode. Auch für diese Größe ist ein Plateau erreicht, hier ab ca. \SI{7}{\milli\meter}. Für die Plattenelektrode ist ab \SI{6}{\milli\meter} ein solches Plateau erreicht, ab dem die Mittelwerte um \SI{10}{\kilo\volt} schwanken. Neben der Durchbruchspannung sind in \figref{fig:box-gap-event-voltage-diff-stab} und \ref{fig:box-gap-event-voltage-diff-plate} die Verteilungen der Differenz zwischen Endspannung und Startspannung (Durchbruchspannung) dargestellt. Die Mittelwerte der Spannungsdifferenzen liegen zwischen \SIrange{1}{10}{\kilo\volt} sowohl für die Stabelektrode als auch für die Plattenelektrode. Wie bei der Durchbruchspannung steigt die Spannungsdifferenz für die ersten beiden Spaltabstände und erreicht danach ein Plateau. Ebenfalls analog zu den Spaltabständen \SI{6}{\milli\meter}, \SI{10}{\milli\meter} und \SI{11}{\milli\meter} liegen diese Ausreißer näher an \SI{1}{\kilo\volt} als an \SI{10}{\kilo\volt}. Dies ist wenig verwunderlich, da kleinere Durchbruchspannungen und damit kleinere Startspannungen nur niedrigere Endspannungen zulassen und somit kleinere Spannungsdifferenzen.

\begin{figure}[H]
  \centering
  \begin{subfigure}[t]{0.48\textwidth}
    \centering
    \includegraphics[width=\textwidth]{figures/classes/voltage_box_electrodes_gap.pdf}
    \caption{Die Durchbruchspannung des Events aufgetragen gegen den Spaltabstand d für die Stabelektrode}
    \label{fig:box-gap-event-voltage-stab}
  \end{subfigure}
  \begin{subfigure}[t]{0.48\textwidth}
    \centering
    \includegraphics[width=\textwidth]{figures/classes/voltage_box_plates_gap.pdf}
    \caption{Die Durchbruchspannung des Events aufgetragen gegen den Spaltabstand d für die Plattenektrode}
    \label{fig:box-gap-event-voltage-plate}
  \end{subfigure}
  \centering
  \begin{subfigure}[t]{0.48\textwidth}
    \centering
    \includegraphics[width=\textwidth]{figures/classes/voltage_drop_box_electrodes_gap.pdf}
    \caption{Die Spannungsdifferenz des Events aufgetragen gegen den Spaltabstand d für die Stabelektrode}
    \label{fig:box-gap-event-voltage-diff-stab}
  \end{subfigure}
  \begin{subfigure}[t]{0.48\textwidth}
    \centering
    \includegraphics[width=\textwidth]{figures/classes/voltage_drop_box_plates_gap.pdf}
    \caption{Die Spannungsdifferenz des Events aufgetragen gegen den Spaltabstand d für die Plattenektrode}
    \label{fig:box-gap-event-voltage-diff-plate}
  \end{subfigure}
  \caption{Verteilung der Durchbruchspannung und der Spannungsdifferenz}
  \label{fig:box-gap-event-voltage-voltage-diff}
\end{figure}


\subsection{Event-Länge}
Für die Stabelektrode liegen die Mittelwerte der Eventlängen zwischen \SIrange{2}{10}{\micro\second}, wobei \SI{10}{\milli\meter} erneut einen Ausreißer mit einem Mittelwert von \SI{20}{\micro\second} darstellt. Die Betrachtung der Plattenelektrode \figref{fig:box-gap-event-duration-plate} zeigt, dass die Mittelwerte der Eventlängen im Intervall \SIrange{10}{20}{\micro\second} liegen. Die \SI{3}{\milli\meter} stellen erneut einen Ausreißer dar; mit \SI{300}{\micro\second} liegt dieser Mittelwert deutlich über den anderen. Dies ist darauf zurückzuführen, dass für einen großen Anteil dieser Spaltabstände eine niedrige zeitliche Auflösung gewählt wurde und dieser Wert daher der Makrostruktur des Events entspricht. In \tabref{tab:timelen-plate-3} sind die Zeitlängen und die Anzahl der Ereignisse für die Plattenelektrode mit einem Spaltabstand von \SI{3}{\milli\meter} eingetragen. Es ist zu erkennen, dass, wie vermutet, der größte Anteil der Entladungen im Bereich von \SIrange{2}{200}{\milli\second} liegt.

\begin{figure}[H]
  \centering
  \begin{subfigure}[t]{0.48\textwidth}
    \centering
    \includegraphics[width=\textwidth]{figures/classes/event_duration_box_electrodes_gap.pdf}
    \caption{Die Eventlänge des Events aufgetragen gegen den Spaltabstand d für die Stabelektrode}
    \label{fig:box-gap-event-duration-stab}
  \end{subfigure}
  \begin{subfigure}[t]{0.48\textwidth}
    \centering
    \includegraphics[width=\textwidth]{figures/classes/event_duration_box_plates_gap.pdf}
    \caption{Die Eventlänge des Events aufgetragen gegen den Spaltabstand d für die Plattenektrode}
    \label{fig:box-gap-event-duration-plate}
  \end{subfigure}
  \caption{Verteilung der Eventlänge}
  \label{fig:box-gap-event-duration}
\end{figure}


\begin{table}[h!]
\centering
\begin{tabular}{|c|c|}
\hline
\textbf{Zeitlänge} & \textbf{Anzahl} \\
\hline
0.0099999   & 138 \\
0.00199998  & 73  \\
9.9999e-05  & 34  \\
0.000799992 & 27  \\
0.199998    & 18  \\
\hline
\end{tabular}
\caption{Anzahl der Zeitlängen der Entladungen für die Plattenelektrode für den Spaltabstand \SI{3}{\milli\meter}}
\label{tab:timelen-plate-3}
\end{table}

