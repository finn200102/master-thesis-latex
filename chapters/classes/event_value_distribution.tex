\section{Verteilung der Eventeigenschaften}
\label{sec:event_value_distribution}

\subsection{Event Ladung und Energie}
\label{sec:event_charge_energie}

\begin{figure}[H]
  \centering
  \begin{subfigure}[t]{0.48\textwidth}
    \centering
    \includegraphics[width=\textwidth]{figures/classes/event_charge_box_electrodes_gap.pdf}
    \caption{Die Ladung des Events aufgetragen gegen den Spaltabstand d für die Stabelektrode}
    \label{fig:box-gap-eventcharge-stab}
  \end{subfigure}
  \begin{subfigure}[t]{0.48\textwidth}
    \centering
    \includegraphics[width=\textwidth]{figures/classes/event_charge_box_plates_gap.pdf}
    \caption{Die Ladung des Events aufgetragen gegen den Spaltabstand d für die Plattenektrode}
    \label{fig:box-gap-event-charge-plate}
  \end{subfigure}
  \centering
  \begin{subfigure}[t]{0.48\textwidth}
    \centering
    \includegraphics[width=\textwidth]{figures/classes/event_energy_box_electrodes_gap.pdf}
    \caption{Die Energie des Events aufgetragen gegen den Spaltabstand d für die Stabelektrode}
    \label{fig:box-gap-event-energy-stab}
  \end{subfigure}
  \begin{subfigure}[t]{0.48\textwidth}
    \centering
    \includegraphics[width=\textwidth]{figures/classes/event_energy_box_plates_gap.pdf}
    \caption{Die Energie des Events aufgetragen gegen den Spaltabstand d für die Plattenektrode}
    \label{fig:box-gap-event-energy-plate}
  \end{subfigure}
  \caption{Verteilung der Event Ladung und der Energie}
  \label{fig:box-gap-event-charge-energy}
\end{figure}

\subsection{Event Durchbruchspannung und Spannungsdifferenz}

\begin{figure}[H]
  \centering
  \begin{subfigure}[t]{0.48\textwidth}
    \centering
    \includegraphics[width=\textwidth]{figures/classes/voltage_box_electrodes_gap.pdf}
    \caption{Die Durchbruchspannung des Events aufgetragen gegen den Spaltabstand d für die Stabelektrode}
    \label{fig:box-gap-event-voltage-stab}
  \end{subfigure}
  \begin{subfigure}[t]{0.48\textwidth}
    \centering
    \includegraphics[width=\textwidth]{figures/classes/voltage_box_plates_gap.pdf}
    \caption{Die Durchbruchspannung des Events aufgetragen gegen den Spaltabstand d für die Plattenektrode}
    \label{fig:box-gap-event-voltage-plate}
  \end{subfigure}
  \centering
  \begin{subfigure}[t]{0.48\textwidth}
    \centering
    \includegraphics[width=\textwidth]{figures/classes/voltage_drop_box_electrodes_gap.pdf}
    \caption{Die Spannungsdiffernz des Events aufgetragen gegen den Spaltabstand d für die Stabelektrode}
    \label{fig:box-gap-event-voltage-diff-stab}
  \end{subfigure}
  \begin{subfigure}[t]{0.48\textwidth}
    \centering
    \includegraphics[width=\textwidth]{figures/classes/voltage_drop_box_plates_gap.pdf}
    \caption{Die Spannungsdiffernz des Events aufgetragen gegen den Spaltabstand d für die Plattenektrode}
    \label{fig:box-gap-event-voltage-diff-plate}
  \end{subfigure}
  \caption{Verteilung der Durchbruchspannung und der Spannungsdiffernz}
  \label{fig:box-gap-event-voltage-voltage-diff}
\end{figure}


\subsection{Event Länge}

\begin{figure}[H]
  \centering
  \begin{subfigure}[t]{0.48\textwidth}
    \centering
    \includegraphics[width=\textwidth]{figures/classes/event_duration_box_electrodes_gap.pdf}
    \caption{Die Eventlänge des Events aufgetragen gegen den Spaltabstand d für die Stabelektrode}
    \label{fig:box-gap-event-duration-stab}
  \end{subfigure}
  \begin{subfigure}[t]{0.48\textwidth}
    \centering
    \includegraphics[width=\textwidth]{figures/classes/event_duration_box_plates_gap.pdf}
    \caption{Die Eventlänge des Events aufgetragen gegen den Spaltabstand d für die Plattenektrode}
    \label{fig:box-gap-event-duration-plate}
  \end{subfigure}
  \caption{Verteilung der Eventlänge}
  \label{fig:box-gap-event-duration}
\end{figure}














%In den Abbildungen \ref{fig:violin-gap-charge-mint} bis \ref{fig:violin-gap-vdiff-maxt} werden die Größen Ladung \(Q\), Eventlänge, maximale Strom \(I_{max}\) und die Spannungsdiffernz \(U_{diff}\) gegen das Produkt aus dem Druck \(p\) und dem Spaltabstand \(d\) aufgetragen. Die linken Plots entsprechen einer niedrigen zeitlichen Auflösung also ohne Mikrostruktur der Entladung und die rechten einer hohen Auflösung mit Mikrostruktur. In \figref{fig:violin-gap-charge-mint} ist ein Trend für die Ladung von \(\SI{1e-8}{\coulomb}\) nach \(\SI{1e-9}{\coulomb}\) zuerkennen, also für niedrigere \(pd\) wurden höhere Gesamtladungsübertragungen beobachtet. Hierbei ist zu beachten, dass dies nicht ebenfalls in \figref{fig:violin-gap-charge-maxt} auftritt, dies ist jedoch nicht verwunderlicht eine hohe zeitliche Auflösung führt dazu, dass nicht mehr das gesammte Event abgebildet werden kann und somit auch nicht die gesammte Ladung erfasst wird. 
%Die Betrachtung von \figref{fig:violin-gap-eventlen-mint} zeigt, dass sich auch hier ebenfalls der Bereich bei \(d = \SI{1}{\centi\meter}\) sich von den anderen unterscheidet. Für die restlichen Werte lässt sich kein klarer Trend für die Eventlänge erkennen. Der Vergleich mit \figref{fig:violin-gap-eventlen-maxt} zeigt, dass für eine hohe Zeitauflösungen, wo nicht das gesamte Event aufgelöst werden kann, auch die Eventzeit kürzer ist. Hierbei ist zu beachten, dass bei niedrigen zeitlichen Auflösungen auch weitere Effekte neben der mikroskopischen Entladung sicht bar werden, wie die parasitären Kapazitäten des restlichen Aufbaus.

%In den Abbildungen \ref{fig:violin-gap-vdiff-mint} und \ref{fig:violin-gap-vdiff-maxt} ist ein klarer Trend von kleinen Spannungsdifferenzen zwischen der Entladungsspannung und der Endspannung von \(\SIrange{0}{10}{kV}\) zuerkennen. Dieser Trend liegt für beide Zeitauflösungen vor ist jedoch für die niedrige Auflösung mit Makrostrukture stärker bzw. es liegen für \(\SIrange{2}{5}{\torr\cdot\centi\meter}\) die Mittelwerte näher an \(\SI{10}{kV}\). Dies ist überraschend, da bei der niedrigen zeitlichen Auflösung die Entladung vollständig vorliegt und somit auch eine nahezu vollständige Aufladung wieder erwarten werden könnte bis zum Ende der Aufnahme. Jedoch in \figref{fig:violin-gap-eventlen-mint} sind kürzere Eventlängen für dieses Interval von \(pd\) zu sehen.


