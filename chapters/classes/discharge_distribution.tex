\section{Beobachtete Entladungstypen}
Für die aufgenommenen Entladungen liegen der Druck \(p\) sowie der Spaltabstand \(d\) vor. Damit lässt sich mit der Entladungsspannung \(U_B\) das Paschen-Gesetz darstellen. In \figref{fig:paschencurvestab} ist die Paschenkurve für die Stabelektrode und in \figref{fig:paschencurveplate} jene für die Plattenelektrode gezeigt. Es ist zu erkennen, dass sich nicht eine einzelne Paschenkurve ergibt, auf der alle Punkte liegen, sondern mehrere horizontal verschobene Kurven. Die Einfärbung der Punkte zeigt, dass sich die einzelnen Teilkurven durch den bei der Messung verwendeten Spaltabstand unterscheiden. Dieser Unterschied zwischen den Kurven bei verschiedenen Spaltabständen ist nicht verwunderlich, da, wie in \secref{sec:paschenlaw} erläutert, \eqref{eq:townsendbreak} vernachlässigt, dass das Feld nicht für alle Elektrodengeometrien gleichförmig ist, sondern insbesondere eine Änderung des Spaltabstands \(d\) zu einer Änderung des Feldes führt und somit zu einer anderen Kurve. Ein Vergleich mit \figref{fig:paschencurve} zeigt, dass nur die rechte Hälfte der Paschenkurve wiedergegeben wurde, die Messwerte von \(pd\) also rechts vom Paschenminimum liegen. Für den Spaltabstand \(\SI{1}{\milli\meter}\) zeigen sich zwei Kurven. Diese könnten als linke und rechte Hälfte der Paschenkurve interpretiert werden; dies entspräche jedoch nicht der Form der Paschenkurve aus \figref{fig:paschencurve}. Wahrscheinlicher ist, dass sich durch mehrere Messungen, zwischen denen dieser Spaltabstand verändert wurde, zwei unterschiedliche Geometrien ergeben haben. Neben der unskalierten Paschenkurve wurde eine skalierte Version mit einem geometrischen Faktor dargestellt. Dieser lautet für die Plattenelektroden \(\frac{A_{\mathrm{elec}}}{d^2}\) und für die Stabelektrode vereinfachend \(\frac{1}{d^2}\). Es zeigt sich, dass ein zu \(\frac{1}{d^2}\) proportionaler Skalierungsfaktor nicht zu einer einzelnen Paschenkurve führt. Die Nichtlinearität des elektrischen Feldes müsste daher weiter untersucht werden, um die geeignete Skalierung zu finden, welche die einzelnen Kurven zu einer vereinigt.


\begin{figure}[H]
  \centering
  \begin{subfigure}{0.48\textwidth}
    \centering
    \includegraphics[width=\textwidth]{../figures/classes/paschen_electrodes.pdf}
    \caption{Original}
  \end{subfigure}
  \hfill
  \begin{subfigure}{0.48\textwidth}
    \centering
    \includegraphics[width=\textwidth]{../figures/classes/paschen_electrodes_scaled.pdf}
    \caption{Skaliert}
  \end{subfigure}
  \caption{Paschenkurve für die Stabelektrode}
  \label{fig:paschencurvestab}
\end{figure}

\begin{figure}[H]
  \centering
  \begin{subfigure}{0.48\textwidth}
    \centering
    \includegraphics[width=\textwidth]{../figures/classes/paschen_plates.pdf}
    \caption{Original}
  \end{subfigure}
  \hfill
  \begin{subfigure}{0.48\textwidth}
    \centering
    \includegraphics[width=\textwidth]{../figures/classes/paschen_plates_scaled.pdf}
    \caption{Skaliert}
  \end{subfigure}
  \caption{Paschenkurve für die Plattenelektrode}
  \label{fig:paschencurveplate}
\end{figure}
