\section{Entladungsformen}
\label{sec:dischargeforms}
Für die Mikrostruktur der Einzelentladungen ergibt sich eine Vielzahl verschiedener Stromstrukturen. Diese lassen sich grob in drei Kategorien einteilen. In \figref{fig:discharge-long-centered} ist eine Entladung dargestellt, die um und nach dem Event um 0 oszilliert. Es liegt also kein abfallender Strom wie bei der Entladung eines Kondensators vor, sondern nur dieses oszillative Verhalten um \SI{0}{\milli\ampere}. Der erste Teil der Entladung mit den größten Schwingungen und somit den höchsten Strömen dauert etwa \SI{10}{\micro\second} an; ein klares Ende liegt in diesem Beispiel nicht vor, jedoch lässt sich eine Mindest länge von \SI{50}{\micro\second} feststellen.
In \figref{fig:discharge-long-decay} ist eine weitere Kategorie der Mikrostruktur der Entladungen zu erkennen. Hier oszilliert der Strom nicht um 0, sondern um einen exponentiell fallenden Strom. Dieser startet außerhalb des Messbereichs, ist also größer als \SI{0,4}{\milli\ampere}. Auch hier ist ein erster Teil mit den höchsten Strömen und den stärksten Oszillationen zu erkennen; dieser dauert ebenfalls ca. \SI{10}{\micro\second}. Die nächsten \SI{10}{\micro\second} entsprechen ebenfalls noch einem klaren exponentiellen Abfall, jedoch mit kleineren Oszillationen um diesen Strom. Nach diesen \SI{20}{\micro\second} fällt der Strom weiter ab, ist jedoch in den asymptotischen Teil des Entladestroms übergegangen, mit vereinzelten kleineren Peaks und Oszillationen. Auch dieses Event ist nicht vollständig aufgezeichnet, da die Mikrostruktur des Events betrachtet werden soll. Es lässt sich also ebenfalls eine untere Grenze von \SI{50}{\micro\second} setzen. Diese ähnliche Eventlänge dieser beiden Typen deutet auf einen Zusammenhang dieser beiden Entladungsmechanismen hin.
In \figref{fig:discharge-short-centered} ist eine weitere Kategorie der Entladung dargestellt. Diese ist, ähnlich wie die aus \figref{fig:discharge-long-centered}, um \SI{0}{\milli\ampere} zentriert, und ein klassischer Entladungscharakter lässt sich nicht erkennen. Diese Entladung unterscheidet sich in zwei Punkten von der ersten: Zum einen ist die Eventlänge mit ca. \SI{10}{\micro\second} deutlich kürzer, zum anderen ist vor dem Start des Events eine Oszillation mit einem Peak von ca. 50\% des maximalen Stroms zu erkennen. Hierbei ist zu beachten, dass der maximale Strom für diese Messungen abgeschnitten ist und somit dieses Vorevent nicht 50\% des tatsächlichen maximalen Stroms entspricht. Zu dem ist ein solches Vorevent nicht allein diesem Entladungstyp zuzuordnen; auch die oben diskutierten Entladungstypen können ein solches Vorevent besitzen. Dies wird in \chapref{chap:eventstart} ausführlich diskutiert. Der erste Punkt jedoch ist interessant, da diese \SI{10}{\micro\second} Länge des Events der Länge des ersten Teils der in \figref{fig:discharge-long-centered} gezeigten Entladung entspricht und auch der qualitative Verlauf mit diesem Teil der Entladung übereinstimmt. Dies lässt vermuten, dass unter bestimmten Bedingungen das Event nach diesen ersten \SI{10}{\micro\second} endet und keine weiteren Oszillationen mehr auftreten, während für unter anderen Bedingungen, diese Aufteilung könnte auch von statistischer Natur sein, das System weiter schwingt und dies zu einem anscheinend längeren Event führt.

\begin{figure}[H]
    \centering
    \includegraphics[width=\textwidth]{../figures/classes/discharge_long_centered.pdf}
    \caption{Entladung mit gedämpfter Oszilation um 0}
    \label{fig:discharge-long-centered}
\end{figure}

\begin{figure}[H]
    \centering
    \includegraphics[width=\textwidth]{../figures/classes/discharge_long_decay.pdf}
    \caption{Entladung mit gedämpfter Oszilation mit exponentiellen Abklingen}
    \label{fig:discharge-long-decay}
\end{figure}

\begin{figure}[H]
    \centering
    \includegraphics[width=\textwidth]{../figures/classes/discharge_short_centered_high_voltage.pdf}
    \caption{Entladungspuls mit Oszilation um 0}
    \label{fig:discharge-short-centered}
\end{figure}





