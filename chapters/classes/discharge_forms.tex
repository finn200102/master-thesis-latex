\section{Entladungsformen}
Für die Mikrostruktur der einzel Entladungen ergibt sich eine Vielzahl an verschiedenen Strukturen des Stroms. Diese lassen sich grob in drei Kategorien einteilen. In \figref{fig:discharge-long-centered} ist eine Entladung dargestellt, die um und nach dem Event um die 0 oszilliert. Es liegt also kein abfallender Strom, wie bei der Entladung eines Kondensators vor, sondern nur dieses oszillative Verhalten um die \SI{0}{\milli\ampere} vor. Der erste Teil der Entladung mit den größten Schwingungen und somit den größten Strömen dauert etwa \SI{10}{\micro\second} an, ein klares Ende liegt in diesem Beispiel nicht vor, aber eine mindest Länge von \SI{50}{\micro\second} lässt sich feststellen.
In \figref{fig:discharge-long-decay} ist eine weitere Kategorie der Mikrostruktur der Entladungen zu erkennen. Hier oszilliert der Strom nicht um die 0, sondern um einen exponentiell fallenden Strom, dieser startet außerhalb des Messbereiches also ist größer als \SI{0,4}{\milli\ampere}. Auch hier ist ein erster Teil mit den größten Strömen und den stärkesten Oszilationen zuerkennen, dieser dauert ebenfalls ca. \SI{10}{\micro\second}. Die nächsten \SI{10}{\micro\second} entsprechen ebenfalls noch einem klaren exponentiellen Abfall, jedoch mit kleineren Oszilationen um diesen Strom. Nach diesen \SI{20}{\micro\second} fällt der Strom weiter ab, ist jedoch in den asymptotischen Teil des Entladestroms übergegangen, mit vereinzelten kleineren Peaks and Oszillationen. Auch dieses Event ist nicht vollständig aufgezeichnet, da die Mikrostruktur des Events betrachtet werden soll, es lässt sich also ebenfalls eine untere Grenze von \SI{50}{\micro\second} setzen. Diese ähnliche Eventlänge dieser beiden Typen deutet auf einen Zusammenhang dieser beiden Entladungsmechanismen hin.
In \figref{fig:discharge-short-centered} ist eine weitere Kategorie and Entladung dargestellt, diese ist ähnlich wie die aus \figref{fig:discharge-long-centered} um \SI{0}{\milli\ampere} zentriert und kein klassicher Entladungecharaketer lässt sich erkennen. Diese Entladung unterscheidet sich in zwei Punkten von der Ersten. Zum einen ist die Eventlänge mit ca. \SI{10}{\micro\second} deutlich kürzer, zum anderen ist vor dem Start des Events eine Oszillation mit einem Peak von ca. 50\% des maximalen Stroms zuerkennen, hierbei ist zu beachten, dass der maximale Strom für diese Messungen abgeschnitten ist und somit dieses Vorevent nicht 50\% des tatsächlichen maximalen Stroms entspricht. Zum anderen ist ein solches Vorevent nicht alleine diesem Typen an Entladung zu zuordnen, sondern auch die anderen oben diskutierten Entladungestypen können ein solches Vorevent besitzen, dies wird im \chapref{chap:eventstart} auführlich diskutiert. Der erste Punkt jedoch ist interessant, da diese \SI{10}{\micro\second} Länge des Events der Länge des ersten Teils der \ref{fig:discharge-long-centered} Entladung entspricht und auch der qualitative Verlauf mit diesem Teil der Entladung übereinspricht. Somit lässt dies vermuten, dass unter bestimmten Bedingungen, nach diesen ersten \SI{10}{\micro\second} das Event endet und keine weiteren Oszillationen mehr auftreten und für andere Bedingungen, diese Aufteilung könnte auch von statistischer Natur sein, das System weiter schwingt und diese zu diesem anscheinen längeren Event führt.

\begin{figure}[H]
    \centering
    \includegraphics[width=\textwidth]{../figures/classes/discharge_long_centered.pdf}
    \caption{Entladung mit gedämpfter Oszilation um 0}
    \label{fig:discharge-long-centered}
\end{figure}

\begin{figure}[H]
    \centering
    \includegraphics[width=\textwidth]{../figures/classes/discharge_long_decay.pdf}
    \caption{Entladung mit gedämpfter Oszilation mit exponentiellen Abklingen}
    \label{fig:discharge-long-decay}
\end{figure}

\begin{figure}[H]
    \centering
    \includegraphics[width=\textwidth]{../figures/classes/discharge_short_centered_high_voltage.pdf}
    \caption{Entladungspuls mit Oszilation um 0}
    \label{fig:discharge-short-centered}
\end{figure}





