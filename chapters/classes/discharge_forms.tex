\section{Entladungsformen}
%Für die Mikrostruktur der einzel Entladungen ergibt sich eine Vielzahl an verschiedenen Strukturen des Stroms. Diese lassen sich grob in drei Kategorien einteilen. In \figref{fig:discharge-long-1} und in \figref{fig:discharge-long-2} sind zwei Entladungen dargestellt, die nach dem Hauptevent, das nach etwa \SI{10}{\micro\second} vorbei ist, noch weitere Nachschwingungen auftreten, die sich erst langsam über eine Zeit von \SI{100}{\micro\second} abregen. In diesen Fällen schwingt das System mit seiner Eigenresonanz noch weiter. Dieser Typ der Entladung tritt in einem Druchbruchspannungsbereich von \SIrange{5}{12}{\kilo\volt} auf. In \figref{fig:discharge-long-5} und in \figref{fig:discharge-long-6} sind Entladungen zu erkennen, die das klassische Entladungsverhalten eines Kondensator mit dessen exponentiellen abklingen wiederspiegeln. Zudem ist insbesondere für \figref{fig:discharge-long-6} ein Strompeak vor dem Event mit etwa einem Abstand von \SI{10}{\micro\second} zuerkennen, diese werden in \chapref{chap:eventstart} weiter betrachtet. Dieser Typ der Entladung trat bei einer Spannung von \SIrange{15}{20}{\kilo\volt} auf. Desweiteren traten noch weitere etwa \SI{10}{\micro\second} lang andauernde Entladungen mit anderen Formen auf, ohne weiteres Nachschwingungen des Systems, in \figref{fig:discharge-short-2} ist ein solches Event dargestellt, auch hier ist ein Strompeak schon vor dem Event zuerkennen. Neben solch lang andauernden Entladungen gab es auch im niedrigen Spannungsbereich von unter \SI{5}{\kilo\volt} kürzere mit Dauern von ca. \SI{5}{\micro\second}.


\begin{figure}[H]
    \centering
    \includegraphics[width=\textwidth]{../figures/classes/discharge_long_centered.pdf}
    \caption{Entladung mit gedämpfter Oszilation um 0}
    \label{fig:discharge-long-centered}
\end{figure}

\begin{figure}[H]
    \centering
    \includegraphics[width=\textwidth]{../figures/classes/discharge_long_decay.pdf}
    \caption{Entladung mit gedämpfter Oszilation mit exponentiellen Abklingen}
    \label{fig:discharge-long-decay}
\end{figure}

\begin{figure}[H]
    \centering
    \includegraphics[width=\textwidth]{../figures/classes/discharge_short_centered_high_voltage.pdf}
    \caption{Entladungspuls mit Oszilation um 0}
    \label{fig:discharge-short-centered}
\end{figure}





