\chapter{Introduction}
\label{chap:introduction}

\section{Hintergrund und Motivation}
\label{sec:background}

Elektrische Entladungen stellen ein fundamentalles Phänomen in der Hochspannungstechnik und der Plasmaphysik dar. Traditionelle Modelle basieren häufig auf empirischen Gesetzmäßigkeiten wie dem Paschen-Gesetz, das die Durchbruchspannung mit dem Produkt aus Gasdruck und Elektrodenabstand beschreibt. Solche Modelle erfassen jedoch nicht die komplexe, stochastische Natur des Entstehungsprozesses von Entladungen. Insbesondere bleibt unklar, unter welchen konkreten physikalischen Bedingungen und zu welchem Zeitpunkt eine Entladung initiiert wird. Erst durch Experimente mit hoher zeitlicher Auflösung lassen sich die zugrunde liegenden dynamischen Mechanismen und mögliche Vorläuferstrukturen systematisch untersuchen.

\section{Forschungsziele}
\label{sec:objectives}
Das Forschungsziel dieser Arbeit besteht darin, Ereignisse zu identifizieren und zu untersuchen, ob es möglich ist, diese durch Vorläufersignaturen vorher zubestimmen. Also den Zusammenhang zwischen dem Vorzustand und dem Event zu untersuchen. Wenn solche Zusammenhänge mit den zur Verfügung stehenden Mitteln aufgezeichnet werden können, lassen sich darauf aufbauende Methoden entwickeln, die diese Vorläufersignaturen modifizieren bzw. in das System eingreifen, um den Verlauf des Events zu manipulieren bzw. vollständig zu unterbinden. Im Rahmen dieser Arbeiten sollen elektrische Entladungen als Event verwendet werden, da es sich um statistische Prozesse handelt, die sich besonders für eine solche Untersuchung eignen, weil sie im Labor kontrolliert erzeugt und so untersucht werden können. Neben diesen nützlichen Eigenschaften, die sich für eine solche Studie eignen, haben sie auch eine technische Relevanz in der Hochspannungstechnik sowie in der Meterologie für die Vorhersage atmosphärischer Entladungen. Für diese Identifikation und insbesondere für die Untersuchung des Zusammenhangs zwischen den Vorläufersignaturen und dem Event sollen Methoden des maschinellen Lernens verwendet und evaluiert werden, die sich zur Vorhersage elektrischer Entladungen eignen und so auch für allgemeine Events einsetzbar sein sollen. Hierbei werden verschiedene Ansätze zur Merkmalsextraktion und Klassifikation verglichen, um herauszufinden, inwieweit sich physikalische Vorläuferphänomene, wie beispielsweise Streamer, identifizieren und zur Vorhersage verwenden lassen. Der Fokus liegt auf der Kombination experimenteller Daten mit datengetriebenen Modellen, um ein besseres Verständnis der Anfangsphase von Entladungsprozessen zu erlangen.

\section{Zusammenfassung}
\label{sec:intro_summary}
Diese Arbeit ist in drei Hauptabschnitte gegliedert:

Im ersten Abschnitt werden die physikalischen Grundlagen elektrischer Entladungen theoretisch motiviert, der experimentelle Aufbau beschrieben sowie die Datenerfassung und Vorverarbeitung erläutert.

Der zweite Abschnitt bietet einen Überblick über die gewonnenen Messdaten. Es werden die Entladungen klassifiziert und der Beginn der Entladungen detailliert analysiert.

Im dritten Abschnitt werden auf Basis der in der Untersuchung des Starts der Entladung identifizierten Vorläuferphänomene datengetriebene Modelle entwickelt. Ziel ist es, diese Merkmale zur Vorhersage von Entladungen einzusetzen und so ein besseres Verständnis der zugrunde liegenden Dynamik zu gewinnen oder durch eine kleine Störung während des Aufbauprozesses den Verlauf des Events zu manipulieren, sodass die eigentliche Entladung nicht stattfinden kann.
