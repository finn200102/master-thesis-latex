\chapter{Introduction}
\label{chap:introduction}

\section{Hintergrund und Motivation}
\label{sec:background}

Elektrische Entladungen stellen ein fundementelles Phänomen in der Hochspannungstechnik und der Plasmaphysik dar. Traditionelle Modelle basieren häufig auf empireischen Gesetzmäßigkeiten wie dem Paschen Gesetz, welches die Durchbruchspannung mit dem Produkt aus Gasdruck und Elektrodenabstand beschreibt. Solche Modelle erfassen jedoch nicht die komplexe, stochastische Natur des Entstehungsprozesses von Entladungen. Insbesondere bleibt unklar, unter welchen konkreten physikalischen Bedingungen un zu welchem Zeitpunkt eine Entladung initiiert wird. Erst durch Experimente mit hoher zeitlicher Auflösung lassen sich die zugrunde liegenden dynamischen Mechanismen un mögliche Vorläuferstrukturen systematisch untersuchen.

\section{Forschungsziele}
\label{sec:objectives}

Ziel dieser Arbeit ist es, methoden des maschinellen Lernes zu entwickeln und zu evaluieren, die zur Vorhersage elektrischer Entladungen eingesetzt werden können. Es werden hierbei verschiedene Ansätze zur Merkmalsextraktion und Klassifikation verglichen, um herauszufinden, inwieweit sich physikalische Vorläuferphänomene, wie beispielsweise Streamer, identifizieren und zur Vorhersage verwenden lassen. Der Fokus liegt auf der Kombination experimenteller Daten mit datengetriebenen Modellen, um ein besseres Verständnis der Anfangsphase von Entladungsprozessen zu erlangen.

\section{Zusammenfassung}
\label{sec:intro_summary}
Diese Arbeit ist in drei Hauptabschnitte gegliedert:

Im ersten Abschnitt werden die physikalischen Grundlagen elektrischer Entladungen theoretisch motiviert, der experimentelle Aufbau beschrieben sowie die Datenerfassung und vorverarbeitung erläutert. 

Der zweiten Abschnitt bietet einen Überblick über die gewonnen Messdaten. Es die Entladungen klassifiziert und der Beginn der Entladungen detailliert analysiert.

Im dritten Abschnitt werden auf Basis der in der Untersuchung des Startes der Entladung identifizieren Vorläuferphänomene datengetriebene Modelle entwickelt. Ziel ist es, diese Merkmale zur Vorhersage von Entladungen einzusetzen und so ein besseres Verständnis der zugrunde liegenden Dynamik zu gewinnen.
