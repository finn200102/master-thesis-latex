\chapter{Introduction}
\label{chap:introduction}

\section{Hintergrund und Motivation}
\label{sec:background}

Elektrische Entladungen stellen ein fundementelles Phänomen in der Hochspannungstechnik und der Plasmaphysik dar. Traditionelle Modelle basieren häufig auf empirischen Gesetzmäßigkeiten wie dem Paschen Gesetz, welches die Durchbruchspannung mit dem Produkt aus Gasdruck und Elektrodenabstand beschreibt. Solche Modelle erfassen jedoch nicht die komplexe, stochastische Natur des Entstehungsprozesses von Entladungen. Insbesondere bleibt unklar, unter welchen konkreten physikalischen Bedingungen und zu welchem Zeitpunkt eine Entladung initiiert wird. Erst durch Experimente mit hoher zeitlicher Auflösung lassen sich die zugrunde liegenden dynamischen Mechanismen und mögliche Vorläuferstrukturen systematisch untersuchen.

\section{Forschungsziele}
\label{sec:objectives}
Das Forschungsziel dieser Arbeit besteht darin Ereignisse zu identifizieren und diese darin gehend zu untersuchen ob es möglich ist diese durch Vorläufersignaturen vorher zu bestimmen. Also den Zusammenhand der Vorzustandes und dem Event zu untersuchen. Wenn solche Zusammenhänge mit den zur Verfügungstehenden Mitteln gelingen, dann können darauf aufbauende Methoden entwickelt werden, die diese Vorläufersignaturn modifizieren bzw. in das System eingreifen um den Verlauf des Events zu manipulieren bzw. vollständig zu unterbinden. Im Rahmen dieser Arbeiten sollen elektrische Entladungen verwendet werden als Event, da es sich um statistische Prozesse handelt die sich besonders für eine solche Untersuchung eigenen, da sie im Labor kontrolliert erzeugt und so untersucht werden können. Neben diesen nützlichen Eigenschaften die sich für eine solche Studie eignen haben sie auch eine technische Relevanz in der Hochspannungstechnik oder auch in der Meterologie für die Vorhersage von Atmosphärischen Entladungen. Für diese Identifikation und insbesondere für die Untersuchung des Zusammenhanges zwischen den Vorläufersignaturen und dem Event sollen Methoden des maschinellen Lernes verwendet werden und evaluiert werden, die sich zur Vorhersage elektrischer Entladungen eigenen und so auch für allgemeine Events einsetzbar sein sollen. Es werden hierbei verschiedene Ansätze zur Merkmalsextraktion und Klassifikation verglichen um herauszufinden, inwieweit sich physikalische Vorläuferphänomene, wie beispielsweise Streamer, identifizieren und zur Vorhersage verwenden lassen. Der Fokus liegt auf der Kombination experimenteller Daten mit datengetriebenen Modellen, um ein besseres Verständnis der Anfangsphase von Entladungsprozessen zu erlangen.

\section{Zusammenfassung}
\label{sec:intro_summary}
Diese Arbeit ist in drei Hauptabschnitte gegliedert:

Im ersten Abschnitt werden die physikalischen Grundlagen elektrischer Entladungen theoretisch motiviert, der experimentelle Aufbau beschrieben sowie die Datenerfassung und vorverarbeitung erläutert. 

Der zweiten Abschnitt bietet einen Überblick über die gewonnen Messdaten. Es werden die Entladungen klassifiziert und der Beginn der Entladungen detailliert analysiert.

Im dritten Abschnitt werden auf Basis der in der Untersuchung des Startes der Entladung identifizieren Vorläuferphänomene datengetriebene Modelle entwickelt. Ziel ist es, diese Merkmale zur Vorhersage von Entladungen einzusetzen und so ein besseres Verständnis der zugrunde liegenden Dynamik zu gewinnen oder durch eine kleine Störung während dem Aufbauprozess den Verlauf des Events zu manipulieren, so dass die eigentliche Entladunge nicht stattfinden kann.
