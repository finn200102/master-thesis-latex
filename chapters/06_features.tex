\chapter{Datenaufbereitung}
\label{chap:features}

Für die Auswertung der Entladungen werden verschiedene Parameter wie Eventstart, Vorläuferstart, Entladungslänge und weitere benötigt; diese werden in diesem Kapitel definiert.

\section{Eventstart}
\label{sec:eventstart}

Der Start des Events wurde zunächst als der Zeitpunkt definiert, ab dem der Strom eine festgelegte Schwelle überscheritet. Der Wert dieser Schwelle wurde bei der Datenerfassung vor einer längeren Messreihe empirisch durch die Wahl der Triggerschwelle festgelegt. Diese Definition des Eventstarts führte jedoch zu Schwierigkeiten bei Events, bei denen vor dem Maximum des Entladungsstroms bereits kleinere Ströme fließen, also im Fall von Vorläufern bzw. Eventauslösern (Townsend-Lawinen). Das Problem besteht darin, dass der Trigger auch auf diese anspricht. Bei richtiger Wahl des Triggerlevels und gleicher Stromgrößenordnung der Entladungen ist dies unproblematisch; ändert sich jedoch die Stromgröße infolge einer Druckvariation, muss eine andere Definition gewählt werden. Als Eventstart wird daher das Überschreiten von \(65 \%\) des maximalen Stroms der jeweiligen Entladung festgelegt. Dieser empirische Wert hat sich als hinrechend genau erwiesen, da er nicht auf starke Vorläufer anspringt und zugleich bei Events, die nicht direkt mit dem maximalen Strom starten, nicht innerhalb des Events liegt. Dieser Wert ist in \figref{fig:event_markers_auto} dargestellt. Da auch diese Definition nicht für alle Entladungen funktioniert, z. B. bei kleineren, schwächeren und flacheren Entladungen ohne klaren Anstiegspeak, wurde für einen Teil der Events dieser Wert händisch mittels einer Python-GUI gesetzt. Somit lassen sich diese beiden Werte in der späteren Auswertung miteinander vergleichen. Der händische Wert ist in \figref{fig:event_markers_hand} dargestellt.

\section{Eventende}
\label{sec:eventende}
Das Ende des Events ist als der Zeitpunkt definiert, an dem der Strom nach Erreichen des Maximums wieder auf \(10\%\) des ursprünglichen Eventwerts abgesunken ist. Da aufgrund von Oszillationen mehrere Zeitpunkte auftreten können, an denen kurzfristig ein solcher Strom erreicht wird, wird vor der Bestimmung dieses Werts eine Einhüllende des Signals gebildet. Dieser Wert ist in \figref{fig:event_markers_auto} dargestellt; analog zu \secref{sec:eventstart} ist in \figref{fig:event_markers_hand} der händisch gewählte Wert dargestellt.

\section{Eventlänge}
\label{sec:eventlen}
Die Länge des Events wird als Differenz zwischen dem Ende des Events (\ref{sec:eventende}) und dem Start des Events (\ref{sec:eventstart}) definiert. Somit ergeben sich zwei Eventlängen: eine für die automatische Auswahl und eine für die händisch gewählten Werte. 
\begin{figure}[H]
  \centering
  \begin{subfigure}[t]{0.48\textwidth}
    \centering
    \includegraphics[width=\textwidth]{figures/features/event_markers.pdf}
    \caption{Automatisch gesetzte Eventmarker} 
    \label{fig:event_markers_auto}
  \end{subfigure}
  \begin{subfigure}[t]{0.48\textwidth}
    \centering
    \includegraphics[width=\textwidth]{figures/features/event_markers_hand.pdf}
    \caption{Händisch gesetzte Eventmarker}
    \label{fig:event_markers_hand}
  \end{subfigure}
  \caption{Platzierung der Eventmarker}
  \label{fig:event_markers}
\end{figure}


\section{Vorläuferstart}
\label{sec:preeventstart}
Der Start des Vorläufers, sofern ein solcher vorhanden ist, wird über \equref{eq:preevent_start} bestimmt. Wenn ab einem Punkt \(i\) in einem Fenster von \(n\) nachfolgenden Punkten mindestens \(p\) Punkte einen Strom größer als die Standardabweichung \(\sigma\) der ersten 1000 Punkte, multipliziert mit einer Schranke \(th\), aufweisen, entspricht \(i\) dem Vorläuferstart. Für die Parameter werden zwei Fälle unterschieden: Ist der maximale Strom \(I_{\text{max}}\) des Events kleiner als \SI{5e-5}{\ampere}, werden \(th = 30\), \(p = 300\) und \(n = 2000\) verwendet; für größere Ströme \(th = 20\), \(p = 10\) und \(n = 2000\). Neben dieser Definition werden ebenfalls händische Werte für einen Teil der Daten erhoben.

\begin{equation}
  \sum_{j=i}^{i+n} \mathbf{1}(I_j > th \cdot \sigma) > p 
  \quad \text{mit} \quad
  \sigma = \sqrt{\frac{1}{1000}\sum_{m=0}^{999} (I_m - \bar{I})^2}
  \label{eq:preevent_start} 
\end{equation}

\subsection{Eventabstand}
\label{sec:event_distance}
Die Differenz aus Eventstart und Vorläuferstart wird als Eventabstand bezeichnet. Sie gibt an, wie lange vor dem Event bereits Vorläuferphänomene auftreten.


\section{Vorläuferende}
\label{sec:preeventend}
Das Ende eines Vorläufers, sofern dieser nicht direkt im Eventstart endet, wird über \equref{eq:precursor_end} definiert. Hierzu wird ab dem Punkt des maximalen Stroms im Zeitfenster zwischen Vorläuferstart und Eventstart das erste Intervall von \(s = 200\) aufeinanderfolgenden Punkten gesucht, für das der Variationskoeffizient kleiner als \(th = 0{,}25\) ist. Vor der Bestimmung dieses Endwerts wird eine Einhüllende des Stroms mit einem Fenster von fünf Punkten gebildet.

\begin{equation}
i_{\text{end}} = \min \left\{ i \, \Bigg| \, 
\begin{aligned}
&  I_{\text{mean}} = \frac{1}{s}\sum_{j=i}^{i+s} I_j \\ 
&  I_{\text{std}} = \sqrt{\frac{1}{s}\sum_{j=i}^{i+s} (I_j - I_{\text{mean}})^2} \\ 
&  \frac{I_{\text{std}}}{I_{\text{mean}}} < th
\end{aligned}
\right\}
\label{eq:precursor_end}
\end{equation}

\begin{figure}[H]
  \centering
  \begin{subfigure}[t]{0.48\textwidth}
    \centering
    \includegraphics[width=\textwidth]{figures/features/precursor_markers.pdf}
    \caption{Automatisch gesetzte Vorläufermarker}
    \label{fig:precursor_markers_auto}
  \end{subfigure}
  \begin{subfigure}[t]{0.48\textwidth}
    \centering
    \includegraphics[width=\textwidth]{figures/features/precursor_markers_hand.pdf}
    \caption{Händisch gesetzte Vorläufermarker}
    \label{fig:precursor_markers_hand}
  \end{subfigure}
  \caption{Platzierung der Vorläufermarker}
  \label{fig:precursor_markers}
\end{figure}

\subsection{Vorläuferabstand}
\label{sec:precursor_distance}
Die Differenz aus Eventstart und Vorläuferende wird als Vorläuferabstand bezeichnet. Diese Größe ist nicht mit dem Eventabstand (\ref{sec:event_distance}) zu verwechseln. Der Vorläuferabstand gibt an, ob der Vorläufer direkt im Eventstart endet oder ob dazwischen ein Intervall mit konstantem, kleinem oder verschwindendem Strom vorliegt. Er entspricht somit auch dem technisch nutzbaren zeitlichen Abstand zwischen Vorläufer und Eventstart, in dem das Intervall von Vorläuferstart bis Vorläuferende ausgewertet und verwendet werden kann.


\section{Durchbruchspannung}
\label{sec:breakdownvoltage}
Die Durchbruchspannung \(U_B\) ist die Spannung unmittelbar vor der Entladung, also die Spannung, die zum Zusammenbruch des Systems geführt hat. Sie entspricht der in \equref{eq:townsendbreak} eingeführten Größe. Sie wird aus den ersten 1000 aufgezeichneten Spannungspunkten durch Mittelwertbildung bestimmt. 


\section{Spannungsdifferenz}
\label{sec:voltage_drop}
Die Spannungsdifferenz \(\Delta U\) beschreibt den Spannungsabfall während der Entladung. Sie wird als Differenz zwischen der Durchbruchspannung \(U_B\) (siehe \secref{sec:breakdownvoltage}) und der nachfolgenden Spannung des Events berechnet:
\begin{equation}
\Delta U = U_B - U_{\text{end}}
\end{equation}
Dieser Parameter gibt Aufschluss über die Energiedissipation während der Entladung und korreliert mit der Intensität des Durchschlags. 

\section{Abgeschnittener Strom}
\label{sec:cutoffcurrent}
Dieser Wert gibt an, ob der maximale Strom mehrfach auftritt. Ist dies der Fall, ist der Strom vertikal abgeschnitten. In \figref{fig:event_voltage_current} ist dies erkennbar. 

\section{Maximaler Strom}
\label{sec:maxcurrent}
Der maximale Strom \(I_{\text{max}}\) ist der Spitzenwert des Entladungsstroms während des gesamten Events. Dieser Parameter dient als Referenzgröße für die Bestimmung des Eventstarts (\(65\%\) von \(I_{\text{max}}\), siehe \secref{sec:eventstart}) und des Eventendes (\(10\%\) von \(I_{\text{max}}\), siehe \secref{sec:eventende}). Der maximale Strom wird direkt aus den aufgezeichneten Stromwerten bestimmt und ist ein Maß für die Intensität der Entladung. In \figref{fig:event_voltage_current} ist dieser als Peak im Stromverlauf erkennbar. Zu beachten ist, dass bei abgeschnittenem Strom der gemessene Maximalwert nicht dem tatsächlichen Maximum entspricht.

\begin{figure}[H]
  \centering
  \includegraphics[width=\textwidth]{figures/features/event_voltage_current.pdf}
  \caption{Visualisierung der Spannungs- und Stromwerte}
  \label{fig:event_voltage_current}
\end{figure}


\section{Eventenergie und Ladung}
\label{sec:eventenergy_charge}
Die während eines Events übertragene Energie und Ladung werden durch Integration der entsprechenden Größen über die Eventlänge bestimmt.

Die übertragene Ladung \(Q\) berechnet sich als Integral des Stroms über die Eventdauer:
\begin{equation}
Q = \int_{t_{\text{start}}}^{t_{\text{end}}} I(t) \, dt \approx \sum_{i=i_{\text{start}}}^{i_{\text{end}}} I_i \cdot \Delta t
\end{equation}

Die dissipierte Energie \(E\) ergibt sich aus dem Produkt von Strom und Spannung:
\begin{equation}
E = \int_{t_{\text{start}}}^{t_{\text{end}}} U(t) \cdot I(t) \, dt \approx \sum_{i=i_{\text{start}}}^{i_{\text{end}}} U_i \cdot I_i \cdot \Delta t
\end{equation}

Hierbei sind \(t_{\text{start}}\) und \(t_{\text{end}}\) der Event- bzw. Vorläuferstart und -ende aus den Abschnitten \ref{sec:eventstart} und \ref{sec:eventende}, und \(\Delta t\) das Abtastintervall der Messung. Diese Parameter ermöglichen eine quantitative Bewertung der Entladungsintensität und sind wichtig für die Klassifizierung verschiedener Entladungstypen.




