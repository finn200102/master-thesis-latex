\section{Aufbau}
\label{sec:setup}
% Für beide Fälle wurde der abfließende Strom über 2 verschiedene Widerstände gemessen, einmal über \SI{100}{\kilo\ohm} und einmal über \SI{200}{\kilo\ohm}. 
Für das Experiment wurde eine Funkenstrecke angefertig inder sich der Abstand der Elektroden über das ein bzw. auschrauben dieser einstellen lässt. Der Kopf der Elektrode lässt sich austauchen um verschiedene Elektroden Formen zu testen. In dieser Arbeit wurden zwei Elektroden verwendet, eine Hutmutter-Elektrode mit einem Radius von \SI{3}{\milli\meter}, die Elektrode ist schematisch in \figref{fig:setup_electrodes} dargestellt und eine Plattenelektrode die in \figref{fig:setup_plate_electrodes} dargestellt ist. Die Plattenelektrode hat eine Kantenlänge von \(\SI{5}{\centi\meter}\), somit also eine Fläche von \(\SI{25}{\square\centi\meter}\).

\begin{figure}[H]
    \centering
    \begin{subfigure}{0.45\textwidth}
        \centering
        \input{figures/setup/setup_electrodes.tex}
        \caption{Zwei runde Elektroden von d = \SI{1}{\centi\metre}.}
        \label{fig:setup_electrodes}
    \end{subfigure}
    \hfill
    \begin{subfigure}{0.45\textwidth}
        \centering
        \input{figures/setup/setup_plate_electrodes.tex}
        \caption{Zwei planare Elektroden von d = \SI{1}{\centi\metre}.}
        \label{fig:setup_plate_electrodes}
    \end{subfigure}
    \caption{Schematische Darstellungen der verwendeten Elektroden.}
    \label{fig:setup_electrodes_combined}
\end{figure}
