\section{Technik}
\label{sec:technic}
Für die Aufnahme der Daten wurde ein Digital Oscilloscope mit \(\SI{800}{\mega\hertz}\) angeschaft, dass DHO4804 von Rigol. Dieses besitzt vier Kanäle. Der erste Kanal wird für die Messung des abfließenden Stromes hinzur Erde von der Funkenstrecke verwendet, der zweite Kanal für die Aufnahme der Spannung an der Funkenstrecke, der dritte Kanal zur Aufnahme des Ladetroms zwischen dem Netzteil und der Funkenstrecke und der vierte Kanal zur Aufnahme der elektromagnetischen Strahlung in ca. \(\SI{50}{\centi\meter}\) Entfernung von der Funkenstrecke. Die Messung des abfließenden Strom wird über die Messung einer Testbox die zwischen die Erde und der Erdseite des Funkenstrecke geschlossen ist, von dieser wird über Koxialkabel dies direkt an Kanal 1 angeschlossen, die Testboxen werden weiter in \secref{sec:messbox} behandelt. Die Messung der Hochspannung an der Funkenstrecke erfolgt über einen Spannungsteiler dieser wird weiter in \secref{sec:voltage_divider} behandelt.

\begin{figure}[H]
    \centering
    \input{figures/setup/setup_circuit.tex}
    \caption{Schaltplan des Versuchaufbaus}
    \label{fig:setup_circuit}
\end{figure}


\subsection{Schwingkreis}

\subsubsection{LC-Resonanz}
Ein System aus einer Induktivität \(L\), einer Kapazität \(C\) und einem Widerstand \(R\) bilden einen gedämpften Schwingkreis. Die Eigenfrequenz \(f_0\) ist

\begin{equation}
f_0 = \frac{1}{2\pi \sqrt{LC}}
\end{equation}

\subsubsection{Gedämpfter RLC-Schwingkreis}
In realen Systemem wie den gemessenen Entladungen müssen die sowohl die \(RLC\) Anteile der Komponenten wie der Funkenstrecke berücksichtigt werden also auch die parasitären Effekte der verwendeten Kabel. Ein solcher gedämpfter \(RLC\)-Kreis wird durch die folgende Differentialgleichung beschrieben:

\begin{equation}
L\frac{d^2q}{dt^2} + R\frac{dq}{dt} + \frac{q}{C} = 0
\end{equation}
Der Fall geringer Dämpfung, also mit längeren oszillatorischen Effekten, ist interessant für die Betrachtung von Entladungen, wie sie in der Hochspannungstechnik auftreten. Die Fälle von kritischer oder aperiodischer Dämpfung klingen ohne längere Osilationen ab.
\subsubsection{Qualitätsfaktor}
Der Qualitätsfaktor \(Q\) gibt das Verhältnis aus gespeicherter und dissipierter Energie an:
\begin{equation}
Q = \frac{1}{R} \sqrt{\frac{L}{C}}
\end{equation}
Hohe Werte entsprechen niedrigen Verlusten des Schwingkreises und niedrige Werte entsprechen einer stärkeren Dämpfung und somit stärkeren Verslusten.

\subsubsection{Spannungsteiler}
\label{sec:voltagedividertheo}
Um hohe Spannungen zumessen muss deren Spannung auf ein niedrigers Level reduziert um die empfindlichen Komponenten der Messelektronik zu schützen. Diese Aufgabe übernimmt ein Spannungsteiler, dieser wird über die Kombination aus zwei Widerständen realisiert, die Ausgangsspannung wird folgenderweise bestimmt
\begin{equation}
    U_{out} = U_{in} \frac{R_2}{R_1 + R_2}
    \label{eq:voltagedivider}
\end{equation}
Hierbei ist \(R_2\) der kleinere Widerstand über den gemessen wird und \(R_1\) der größere Widerstand. Nicht nur das Verhältnis ist relevant, sonder auch der totale Widerstand in Reihe \(R = R_1 + R_2\) da bei niedrigen Widerständen ein großer Strom fließen würde, somit müssen genügend große gewählt werden. Unter Gleichspannung muss nichts weiter beachtet werden, jedoch für Wechselspannung wird das Verhalten wegen limitierender parasitären Effekte beinträchtigt. Für große Widerstände liegt jedoch auch eine hohe Impedanz für das Element vor. Die Ausgangsimpendanz lässt sich folgender weise berechenen:
\begin{equation}
    Z_{out} = \frac{R_1 R_2}{R_1 + R_2}
    \label{eq:voltagedividerimp}
\end{equation}
Somit ergibt sich für große \(Z_out\) ein Tiefpassfilter. Neben der Ausgangsimpendanz des Spannungsteilers müssen ebenfalls die kapazitiven parasitären Effekte betrachtet werden, also die Kapazitivität des Intrumentes und der verbundenen Kabel.
\begin{equation}
    f_c = \frac{1}{2\pi Z_out C_in}
    \label{eq: voltagedividerfreq}
\end{equation}
Somit ist die Bandbreite der Spannungsmesung mit einem Spannungsteiler auf diese Bandbreite begrenzt. Anteile die über dieser Frequenz liegen werden verstärkt und verschoben und führen so zu Ungenauigkeiten. 

\subsection{Spannungsteiler}
\label{sec:voltage_divider}
Für die Messung der Spannung des Netzteils, also der Spannung an der Funkenstrecke wurde ein Spannungsteiler aus einem \SI{100}{\mega\ohm} und einem \SI{100}{\kilo\ohm} Widerstand angefertigt. In \figref{fig:setup_voltage_divider} ist ein Schaltplan für diesen Spannungteiler. Aus \equref{eq:voltagedivider} lässt sich somit die das Verhältnis des Spannungsteilers bestimmen, es ergibt sich eine Reduktion der Spannung um den Faktor 1001. Es wird im folgenden vereinfachend der Wert 1000 verwendet.

\begin{figure}[H]
    \centering
    \resizebox{0.5\textwidth}{!}{%
        \input{figures/setup/setup_voltage_divider.tex}
    }
    \caption{Schaltplan des Spannungsteilers}
    \label{fig:setup_voltage_divider}
\end{figure}


\subsection{Messboxen}
\label{sec:messbox}
Für die Messung des abfließenden Stroms von der Funkenstrecke hinzur Erde, wurde eine Messbox verwendet, in der ein über ein Widerstand gemessen wurde. Um die Spannung zubegrenzen um mit großen Widerständen kleine Ströme sicher zu messen wurden zwei Dioden verwendet, die anti paralel paralel zum Widerstand angebrach wurden. In \figref{fig:setup_box} wird der Schaltplan einer solchen Box dargestellt. Es wurden drei Boxen angefertigt mit verschiedenen Konfigurationen. In \tabref{tab:boxsetup} Sind die diese verwendeten Konfigurationen angegeben.

\begin{figure}[H]
    \centering
    \resizebox{0.5\textwidth}{!}{%
        \input{figures/setup/box.tex}
    }
    \caption{Schaltplan der Messbox}
    \label{fig:setup_box}
\end{figure}

\begin{table}[H]
\centering
\caption{Konfigurationen der Messboxen}
\label{tab:boxsetup}
\begin{tabular}{S[table-format=2.0] S[table-format=2.1] S[table-format=2.1]}
\toprule
\textbf{Box} & \textbf{Widerstand R [\si{\kilo\ohm}]} & \textbf{Diode [\si{\kilo\volt}]}\\
\midrule
A  & 100 & \text{P6KE7.6A} \\
B  & 200 & \text{P6KE7.6A}\\
C  & 100 & \text{UF407}\\
\bottomrule
\end{tabular}
\end{table}


