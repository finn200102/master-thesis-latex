\chapter{Eventstart}
\label{chap:eventstart}

\section{Eventabstände}
\label{sec:eventdistance}

Der in \secref{sec:event_distance} definierte Abstand ist für die Entladungen nach \figref{fig:highres-discharge} in \figref{fig:histogram-eventdistance} für die automatische Definition der Eventgrenzen und in \figref{fig:histogram-eventdistance-hand} für die händisch gesetzten Grenzen dargestellt. In \ref{fig:histogram-eventdistance} sind vier Bereiche ersichtlich, zwei die an den Grenzen liegen, bei \SI{1e-9}{\second} und bei \SI{1e-5}{\second}. Diese beiden Regionen entsprechen für den kleinen Eventabstände garnicht bzw. kaum vorhandenen Vorevents, also dem Fall, dass Eventstart gleich Vorläuferstart ist und den großen Eventabständen bei denen die automatische Bestimmung der Vorläuferstartes zu sensitiv auf frühe Schwankungen ist. Die beiden mittleren Intervalle von \SIrange{2e-9}{3e-8}{\second} und von \SIrange{3e-8}{1e-5}{\second}. Das erste Intervall entspricht kurzen Vorläuferphänomenen, bei denen die meißten direkt am Eventstart hängen und entsprechen für die kürzesten Oszillationen, die direkt auf dem exponentiellen Anstieg des Events liegen und für die etwas längeren Oszillationen, die noch kurz vor dem exponentiellen Eventstart liegen. Im zweiten Intervall liegen sowohl weitere flache Oszillationen die wie im ersten Intervall direkt vor dem Event liegen aber auch Oszillationen mit einem stärkeren Peak nach dem ein konstanter kleiner Strom fließt, dieser hat Streamercharakter. Der Vergleich der automatisch gesetzten Grenzen mit den händischen in \figref{fig:histogram-eventdistance-hand} zeigt zum einen, dass weniger Entladungen händisch annotiert wurden, aber auch dass nur eine Verteilung vorliegt, die einen normalverteilten Charakter aufweist, mit einen Schweif nach rechts. Diese Verteilung entspricht dem dritten Bereich der automatisch gesetzen Grenzen, da für die händische Annotation nur die längeren besonders Interessanten Vorevents gewählt wurden. In den Abbildungen \ref{fig:box-eventdistance-electrode} bis \ref{fig:box-eventdistance-plate-hand} sind die Eventabstände für die beiden Elektrodengeometrien und für die aautomatische und händische Version aufgetragen gegen den Spaltabstand. Es ist zubeachten, dass die y-Axis der Plots also der Eventabstand logarithmisch skalliert ist. Somit entspricht der klare lineare Trend einem exponentiellen Trend von kleinen Spaltabständen hin zu großen Spaltabständen für den Eventabstand. Aufgrund der geringeren Datenmenge und der selektiven Auswahl dieser ist dieser Trend weniger gut erkennbar und schwächer ausgeprägt für die händisch gesetzten Grenzen, jedoch in \figref{fig:box-eventdistance-plate-hand} trotzdem erkennbar.







\begin{figure}[htbp]
    \centering
    \begin{minipage}[t]{0.47\textwidth}
      \centering
      \includegraphics[width=\linewidth]{../figures/event_start/event_distance.pdf}
      \caption{Verteilung der Eventabstände}
      \label{fig:histogram-eventdistance}
    \end{minipage}
 \begin{minipage}[t]{0.47\textwidth}
      \centering
      \includegraphics[width=\linewidth]{../figures/event_start/event_distance_hand.pdf}
      \caption{Verteilung der Eventabstände (händisch)}
      \label{fig:histogram-eventdistance-hand}
  \end{minipage}
\end{figure}

\begin{figure}[htbp]
    \centering
    \begin{minipage}[t]{0.47\textwidth}
      \centering
      \includegraphics[width=\linewidth]{../figures/event_start/event_distance_box_electrodes.pdf}
      \caption{Spaltabstand d gegen Eventabstände (Elektrode)}
      \label{fig:box-eventdistance-electrode}
   \end{minipage}
 \begin{minipage}[t]{0.47\textwidth}
      \centering
      \includegraphics[width=\linewidth]{../figures/event_start/event_distance_box_plates.pdf}
      \caption{Spaltabstand d gegen Eventabstände (Platte)}
      \label{fig:box-eventdistance-plate}
  \end{minipage}
\begin{minipage}[t]{0.47\textwidth}
      \centering
      \includegraphics[width=\linewidth]{../figures/event_start/event_distance_manual_box_electrodes.pdf}
      \caption{Spaltabstand d gegen Eventabstände (Elektrode, händisch)}
      \label{fig:box-eventdistance-electrode-hand}
   \end{minipage}
 \begin{minipage}[t]{0.47\textwidth}
      \centering
      \includegraphics[width=\linewidth]{../figures/event_start/event_distance_manual_box_plates.pdf}
      \caption{Spaltabstand d gegen Eventabstände (Platte, händisch)}
      \label{fig:box-eventdistance-plate-hand}
  \end{minipage}

\end{figure}

\subsection{Vorläuferabstände}
\label{sec:precursor_distance_target}
Die Betrachtung von \figref{fig:histogram-precdistance} und der Vergleich mit \figref{fig:histogram-eventdistance} zeigt, dass sowohl die beiden Grenzbereiche nicht mehr vorliegen. Der linke fällt weg, da kein Vorläuferende für solch kurze Phänomene auftritt und das rechte fällt weg, da diese großen Abstände Fehlern in dem Grenzsetzung entsprechen. Ebenfall ist erkenntlich, dass nur einem deutlich kleinerer Teil der Entladungen eine Ende zugeordnet werden konnte. Und dass diese Reduktion der Daten für die beiden Intervalle unterschiedlich stark erfolgte und somit das Verhältnis dieser beiden Bereich ausgetauscht hat.


\begin{figure}[htbp]
    \centering
    \begin{minipage}[t]{0.47\textwidth}
      \centering
      \includegraphics[width=\linewidth]{../figures/event_start/precursor_distance.pdf}
      \caption{Verteilung der Vorläuferabstände}
      \label{fig:histogram-precdistance}
   \end{minipage}
 \begin{minipage}[t]{0.47\textwidth}
      \centering
      \includegraphics[width=\linewidth]{../figures/event_start/precursor_distance_hand.pdf}
      \caption{Verteilung der Vorläuferabstände (händisch)}
      \label{fig:histogram-precdistance-hand}
  \end{minipage}
\end{figure}

\begin{figure}[htbp]
    \centering
    \begin{minipage}[t]{0.47\textwidth}
      \centering
      \includegraphics[width=\linewidth]{../figures/event_start/precursor_distance_box_electrodes.pdf}
      \caption{Spaltabstand d gegen Vorläuferabstände (Elektrode)}
      \label{fig:box-precdistance-electrode}
   \end{minipage}
 \begin{minipage}[t]{0.47\textwidth}
      \centering
      \includegraphics[width=\linewidth]{../figures/event_start/precursor_distance_box_plates.pdf}
      \caption{Spaltabstand d gegen Vorläuferabstände (Platte)}
      \label{fig:box-precdistance-plate}
  \end{minipage}
\begin{minipage}[t]{0.47\textwidth}
      \centering
      \includegraphics[width=\linewidth]{../figures/event_start/precursor_distance_manual_box_electrodes.pdf}
      \caption{Spaltabstand d gegen Vorläuferabstände (Elektrode, händisch)}
      \label{fig:box-precdistance-electrode-hand}
   \end{minipage}
 \begin{minipage}[t]{0.47\textwidth}
      \centering
      \includegraphics[width=\linewidth]{../figures/event_start/precursor_distance_manual_box_plates.pdf}
      \caption{Spaltabstand d gegen Vorläuferabstände (Platte, händisch)}
      \label{fig:box-precdistance-plate-hand}
  \end{minipage}

\end{figure}


\section{Eventstartdynamik}
\label{sec:event_startdynamik}


Die Struktur der vorliegenen Vorläuferphenomene ist im weiteren interessant. In \figref{fig:discharge-vor-klein-1} und in \figref{fig:discharge-vor-klein-2} sind zwei erkannte Vorläuferphenomene für den Bereich \(\SIrange{10}{100}{\nano\second}\) dargestellt, hierbei ist zu erkennen, dass einige (signifikanter Anteil) der erkannten Vorläufer im wesentlichen nur dem exponentiellen Anstieg zu Beginn des Events entsprechen, diese Klasse ist aber trotzdem im Hinblick auf den Eventstart interessant, da hier größere Anstiegszeiten auftreten, da die sonstigen apprupten Anstiege die schneller als \(\SI{10}{\nano\second}\) sind von vorhinein für die Betrachtung von Vorläufern herausgefiltert wurden. In \figref{fig:discharge-vor-klein-2} sind jedoch im Anstieg kleine Oszillationen zuerkennen, also Phänomene die charakteristisch für das Event sein sollten. In den Abbildungen \ref{fig:discharge-vor-mittel-1} bis \ref{fig:discharge-vor-mittel-2} sind zwei Entladungen exemplarisch für den mittleren Bereich von \(\SIrange{0,1}{1}{\micro\second}\) dargestellt. In diesem Bereich treten längere Vorläuferprozesse auf, die analog wie auch in \figref{fig:discharge-vor-klein-2} direkt vor dem Event stattfinden, entweder wie in \figref{fig:discharge-vor-mittel-2} den exponentiellen Anstieg begleiten oder wie in \figref{fig:discharge-vor-mittel-2} wo die Oszillationen eine höhere zweitliche Ausdehnung, höhere Amplituden und schon vor dem exponentiellen Anstiegt stattfinden, also nicht mehr nur begleitend auftreten. In \figref{fig:discharge-vor-lang-1} und \figref{fig:discharge-vor-lang-2} sind zwei Entladungen aus dem letzten Interval für \(s > \SI{1}{\micro\second}\) dargestellt. Neben der in Entladung in \figref{fig:discharge-vor-lang-2} die ähnliche wie \figref{fig:discharge-vor-mittel-1} vor dem Event stattfindet, aber trotzdem direkt im exponentiellen Anstieg des Events endet, ist in \figref{fig:discharge-vor-lang-2} ein weitere Typ and Vorläufer zu erkennen, zumeinem ist der Start diese Vorläufers noch weiter vom Start des Event entfernt, aber der oszillierende Anteil regt sich wieder ab und es bleibt zwischen Vorläufer und Event ein konstanter kleiner Strom stehen, bis es zum Event kommt. Dieser Prozess lässt sich als streamerartig identifizieren, wobei der anfängliche oszillierende Teil sich als die Lawine und die Bildung der Streamerkanäle interpretieren lässt, kann der konstante fließende Strom als der Übergang zwischen Streamer hin zum Leader und der Sparkentladung interpretieren, also der Thermalisierung diese anfänglich nur schwach leitenden Kanals. Gerade diese Vorläufersignaturen die durch einen größeren Abstand zum Event charakterisiert sind und dier erste Bunch an Oszillationen ein Ende aufweist, nach dem dieser Strom konstant bis hin zum Start der Entladungen ist, eignen sich um zu untersuchen inwie fern sich diese Vorläufersignaturen eignen um den Vorläuferabstand zu bestimmen. 

\begin{figure}[H]
    \centering
    \begin{minipage}[t]{0.44\textwidth}
        \centering
        \includegraphics[width=\textwidth, height=0.25\textheight]{../figures/event_start/discharge_short_1.pdf}
        \caption{Entladung für kleine Vorläuferabstände}
        \label{fig:discharge-vor-klein-1}
    \end{minipage}
    \hfill
    \begin{minipage}[t]{0.44\textwidth}
        \centering
        \includegraphics[width=\textwidth, height=0.25\textheight]{../figures/event_start/discharge_short_2.pdf}
        \caption{Entladungen für kleine Vorläuferabstände}
        \label{fig:discharge-vor-klein-2}
    \end{minipage}
\end{figure}



\begin{figure}[H]
    \centering
    \begin{minipage}[t]{0.44\textwidth}
        \centering
        \includegraphics[width=\textwidth, height=0.25\textheight]{../figures/event_start/discharge_middle_1.pdf}
        \caption{Entladung für mittlere Vorläuferabstände}
        \label{fig:discharge-vor-mittel-1}
    \end{minipage}
    \hfill
    \begin{minipage}[t]{0.44\textwidth}
        \centering
        \includegraphics[width=\textwidth, height=0.25\textheight]{../figures/event_start/discharge_middle_2.pdf}
        \caption{Entladungen für mittlere Vorläuferabstände}
        \label{fig:discharge-vor-mittel-2}
    \end{minipage}
\end{figure}

\begin{figure}[H]
    \centering
    \begin{minipage}[t]{0.44\textwidth}
        \centering
        \includegraphics[width=\textwidth, height=0.25\textheight]{../figures/event_start/discharge_long_1.pdf}
        \caption{Entladung für längere Vorläuferabstände}
        \label{fig:discharge-vor-lang-1}
    \end{minipage}
    \hfill
    \begin{minipage}[t]{0.44\textwidth}
        \centering
        \includegraphics[width=\textwidth, height=0.25\textheight]{../figures/event_start/discharge_long_2.pdf}
        \caption{Entladungen für längere Vorläuferabstände}
        \label{fig:discharge-vor-lang-2}
    \end{minipage}
\end{figure}


\section{Eventanstieg}
Der Anstieg des Entladestroms lässt sich durch ein exponentielles Modell beschreiben:
\begin{equation}
    A e^{\frac{t}{\tau}} + C
    \label{eq:current-rise}
  \end{equation}                                                                                  Somit lässt sich aus einem solchen fit die Zeitkonstante \(\tau\) bestimmen und untersuchen wie sich diese unter verschiedenen Bedingungen verändert. In \figref{fig:taus_rise_histo} ist die Verteilung der \(\tau\) dargestellt, es handelt sich um eine Normalverteilung in einem Interval von \SIrange{1e-10}{1e-7}{\second}, wobei kleine Werte von \(\tau\) einer schnelleren Änderung, also einem schnelleren Wachstum des Stroms entsprechen und größere einem langsameren. Neben der Verteilung liegen auf beiden Seiten einmal bei \SI{1e-11}{\second} und bei \SI{1e-3}{\second}. Die \SI{1e-3}{\second} liegen um 5 Größenordnungen außerhalb der Verteilung, so dass diese als Ausreißer behandelt werden, zu solchen Ausreißern kommt es bei besonders schlechten fits, also wenn die Daten z.B alle flach sind, da nur Störsignale vorliegen und so nur ein Wachstum auf eine Größenordnungen von \SI{1}{\milli\second} vorliegt. Die \SI{1e-11}{\second} liegen, jedoch nur um eine Größenordnungen entfernt, so dass hier noch ein Zusammenhand vermutet werden kann, jedoch ligen diese in einer Größenordnungen von \SI{10}{\pico\second} und diese Änderungen sind so schnell, dass mit der Vorhandenen Technik keine klare Aussage über diese mehr getroffen werden kann. Die Normalverteilung erstreckt sich über 3 Größenordnungen von \SI{0,1}{\nano\second} bis \SI{0,1}{\micro\second}. Somit sind die zeitlichen Anstiege des Entladestroms über einen weiten Bereich verteilt. In \figref{fig:taus_rise_box} ist die Verteilung der \(\tau\) für die verschiedenen Spaltabstände dargestellt. In den Boxplots sind auch die deutlichen Ausreißer im \SI{1}{\milli\second} Bereich als solche erkannt worden, wobei die im \SI{10}{\pico\second} Bereich nicht so gekennzeichnet wurde. Die Mittelwerte der \(\tau\) liegen alle zwischen \SIrange{1e-9}{1e-8}{\second}, wobei ein steigender Trend, der aber ein Plato ereicht, vorliegt. Neben dem \(\tau\) sind auch die \(A\) und \(C\) von intresse. Die Verteilung der \(A\) ist in \figref{fig:a_rise_histo} dargestellt, sie folgen ebenfalls eine Normalverteilung die, jedoch nach rechts verschoben ist, die \(A\) liegen in einem Interval von \SIrange{1e-7}{1e-3}{\ampere}. Dieser Wert ist die Amplitude und gibt an, wie groß die Veränderung ist. Die Verteilung der \(C\) is in \figref{fig:c_rise_histo} dargestellt, sie folgt wie auch die \(A\) einer Normalverteilung die nach rechts verschoben ist, die \(C\) liegen in einem Interval von \SIrange{1e-9}{1e-4}{\ampere}. Dieser Wert gibt die y-Verschiebung an, also den Startwert des exponentiellen Wachstums und somit den Strom vor dem Eventstart bzw. beim Start des Eventanstieges. Somit lässt sich aus diesen Werte schließen, dass somit auch die Anfangströme, alle unter \SI{1}{\milli\ampere} Strömen liegen und sich im Mittel in etwa im \SI{1}{\micro\ampere} Bereich liegen oder noch kleiner sind. Die Verteilung der \(A\) zeigt Werte unter 1 im \SI{1}{\micro\ampere} bis \SI{1}{\milli\ampere} Bereich, also spiegelt einfach die Größenordnungen der aufgenommen Ströme wieder, wo hierbei wieder zu beachten ist, dass hier nur die Mikrostruktur der Entladungen untersucht werden und die makroskopsichen großen Ströme nicht aufgezeichnet wurden, die für solche Entladungen im \SI{1}{\ampere} Bereich liegen dürften. Wenn die totale Größe des maximalen Startstromes mit dessen gesammten Anstieg aufgezeichnet wurden wäre, dass würden auch größere Werte für \(A\) vorliegen. Die Werte für diese drei Fitkonstanten, die nicht innerhalb dieser Normalverteilung liegen, können als Fits die auf Daten die nicht diesem exponentiellen Verlauf folgen identifiziert werden. Die Qualität dieses Fits ist in den Abbildungen \ref{fig:score_rise_histo} und \ref{fig:score_rise_histo_zoom} dargestellt, wobei \ref{fig:score_rise_histo_zoom} eine Vergrößerung der Werte die nahe an der 1 liegen, also den fits die sehr gut mit den Daten übereinstimmen, darstellt. Also die Mehrheit der Daten lässt sich gut mit einem solchen exponentiellen Model wie in \eqref{eq:current-rise} erklären. In der \figref{fig:discharge_rise_time_0p8} ist ein solcher Fit mit einem Score von \(0.8\) dargestellt. Es ist zu erkennen, dass entweder die automatische Setzung des Eventstartes fehlerhaft war oder in dieser Messung keine bzw. eine sehr schwache flache Entladung aufgenommen wurde. In \figref{fig:discharge_rise_time_0p9} ist ein besserer Fit mit einem Score von \(0.9\) dargestellt, hier ist das richtige Interval gewählt und der Start dieses Events hat einen klaren exponentiellen Charakter, jedoch ist der Strom zum einen nicht abgeschnitten, was schon auf eine schwächere flachere Entladung hindeutet, sondern auch erkenntlich ist, das der erste Peak des Stroms nicht mal dem maximalen entspricht. Zudem sind die Variationen im Strom vor dem Start des Event hier auch nicht verschwindent klein im Vergleich, zu maximalen Strom und so ergibt sich ein solcher Fit. In \figref{fig:discharge_rise_time} ist ein Score mit einem Wert von über \(0.00\) dargestellt, also eine hervoragende übereinstimmung zwischen den Daten und dem Modell. Dieses starke exponentielle Verhalten liegt vor, da der Strom abgeschnitten ist und so ein sehr starker Anstieg vorliegt.

\begin{figure}[H]                                                                                 \centering
    \begin{minipage}[t]{0.44\textwidth}
        \centering
        \includegraphics[width=\textwidth, height=0.25\textheight]{../figures/event_start/taus_rise_box.pdf}
        \caption{Taus vs Spaltabstand}
        \label{fig:taus_rise_box}                                                               
    \end{minipage}
    \hfill
    \begin{minipage}[t]{0.44\textwidth}
        \centering
        \includegraphics[width=\textwidth, height=0.25\textheight]{../figures/event_start/taus_rise_histo.pdf}
        \caption{Verteilung der Taus}
        \label{fig:taus_rise_histo}
    \end{minipage}
\end{figure}


\begin{figure}[H]
    \centering
    \begin{minipage}[t]{0.44\textwidth}
        \centering
        \includegraphics[width=\textwidth, height=0.25\textheight]{../figures/event_start/a_rise_histo.pdf}
        \caption{Verteilung a}
        \label{fig:a_rise_histo}
    \end{minipage}
    \hfill
    \begin{minipage}[t]{0.44\textwidth}
        \centering
        \includegraphics[width=\textwidth, height=0.25\textheight]{../figures/event_start/c_rise_histo.pdf}
        \caption{Verteilung der c}
        \label{fig:c_rise_histo}
    \end{minipage}
\end{figure}





\begin{figure}[H]
    \centering
    \begin{minipage}[t]{0.44\textwidth}
        \centering
        \includegraphics[width=\textwidth, height=0.25\textheight]{../figures/event_start/score_rise_histo.pdf}
        \caption{Histogram des scores des fits}
        \label{fig:score_rise_histo}
    \end{minipage}
    \hfill
    \begin{minipage}[t]{0.44\textwidth}
        \centering
        \includegraphics[width=\textwidth, height=0.25\textheight]{../figures/event_start/score_rise_histo_zoom.pdf}
        \caption{Histogram des scores des fit zoom}
        \label{fig:score_rise_histo_zoom}
    \end{minipage}
\end{figure}

\begin{figure}[H]
    \centering
    \begin{minipage}[t]{0.32\textwidth}
        \centering
        \includegraphics[width=\textwidth, height=0.25\textheight]{../figures/event_start/discharge_rise_time_0p8.pdf}
        \caption{Fit des Eventstarts mit einem score von \(\text{score}=0.8\)}
        \label{fig:discharge_rise_time_0p8}
    \end{minipage}
    \hfill
    \begin{minipage}[t]{0.32\textwidth}
        \centering
        \includegraphics[width=\textwidth, height=0.25\textheight]{../figures/event_start/discharge_rise_time_0p9.pdf}
        \caption{Fit des Eventstarts mit einem score von \(\text{score}=0.9\)}
        \label{fig:discharge_rise_time_0p9}
    \end{minipage}
    \hfill
 \begin{minipage}[t]{0.32\textwidth}
        \centering
        \includegraphics[width=\textwidth, height=0.25\textheight]{../figures/event_start/discharge_rise_time.pdf}

        \caption{Fit des Eventstarts mit einem score von \(\text{score}=0.99\)}
        \label{fig:discharge_rise_time}
    \end{minipage}

\end{figure}



\section{Eventabfall - mehrere Entladungen}
Das Verhalten des Entladestroms lässt sich über ein einfaches exponentielles Modell beschreiben

\begin{equation}
    Ae^{\frac{-t}{\tau}} + C
    \label{eq:discharge-e}
\end{equation}
Hierbei ist \(\tau\) die Zeitkonstante, die sich nach
\begin{equation}
    \tau = RC
\end{equation} 
in Verbindung mit dem Widerstand R und der Kapazität C bringen lässt. Somit lässt sich C bestimmen wenn R bekannt ist, R lässt sich ebenfalls aus dem Fit bestimmen, aus A. Es gilt \(R = \frac{U_{break}}{A}\). Die Kapazität lässt sich auf drei Wegen bestimmen. Einmal lässt sich R über die Ladung der Entladung bestimmen nach \(C = \frac{Q}{U_{break}}\), einmal nach \(C = \frac{\tau}{\frac{U_{break}}{A}}\) und einmal nach \(C = \frac{\tau}{R}\), wobei hier R dem verwendeten Widerstand gilt der verwendet wurde zum messen des Stroms. 
Die Messungen mit mehreren Entladungen aus \figref{fig:discharge_multiple_fit} lassen sich verwenden um diese Werte zu bestimmen, da somit insbesondere der Mittelwert und die Varianz bestimmt werden kann und somit untersucht werden kann, ob und wie start diese Werte zwischen den Entladungen varieren. In dem dargestellten Plot sind für die einzelnen Entladungen, die fits der Entladeströme nach \eqref{eq:discharge-e} eingetragen. In dem Abbildungen \ref{fig:multiple_histo_tau} bis \ref{fig:multiple_histo_c3} sind die verschiedenen bestimmten Größen dargestellt. Es sind die jeweiligen Statistiken der jeweiligen Größe eingetragen und farblich sind die zeitliche Reihenfolgen der einzelnen Histogrambalken dargestellt, wobei die Gradient von Grün nach Rot geht, was auch der Richtung der Zeit entspricht. Die Betrachtung der eingetragenen fits in \figref{fig:discharge_multiple_fit} zeigt, dass das exponentielle Modell für den Entladestrom (auf dieser makroskopischen Skaala) gut übereinstimmt, jedoch die Form des Entladestroms in zwei Stellen abweicht. Auf der mittleren Höhe des Entladestroms bei ca. \SI{6}{\micro\ampere} ist der Fit steiler und auf der Höhe von \SI{0}{\micro\ampere} ist der Fit flacher als der Strom. Dies liegt daran, dass der erste Strompeak, der nicht vollständig aufgelöst werden kann mit der getroffenen Wahl des Messwiderstandes, den Start des Entladevorgangs für den Fit angibt, jedoch der exponentielle Verlauf erst ein paar \SI{100}{\micro\second} später eintritt. Somit spiegelt das exponentielle Modell nicht vollständig der Verlauf des Entladestroms in diesem Intervall wieder, jedoch stimmt der Verlauf jedoch trotzdem gut genug mit dem simplen exponentiellen Model überein um aus diesen einige Werte aus diesen Entladungen abzuleiten. In \figref{fig:multiple_histo_tau} ist die Verteilung der verschiedenen \(\tau\) dargestellt, der Mittelwert liegt bei \SI{1,26}{\milli\second}. Dieser Wert variert nur gering für die verschiedenen Einzelentladungen mit einer Standartabweichung von \SI{14,1}{\micro\second}. Die zeitliche Dynamik dieser Variation ist jedoch interessant, da sich diese kontinuierlich hin zu größeren \(\tau\) entwickelt, dies ist in dem farblichen Übergang von Grün zu Rot dargestellt. Somit steigt \(\tau\) mit jeder weiteren Entladung unter diesen Bedingungen, also wenn diese direkt auf einander folgen. Dieser Trend ist auch in den Histogrammen für die weiteren Werte erkenntlich.

\begin{figure}[htbp]
    \centering
    \begin{minipage}[t]{\textwidth}
      \centering
      \includegraphics[width=\linewidth]{figures/event_start/discharge_multiple.pdf}
      \caption{Mehrere Entladungen mit fits}
      \label{fig:discharge_multiple_fit}
    \end{minipage}
\end{figure}

\begin{figure}[htbp]
    \centering
    \begin{minipage}[t]{0.47\textwidth}
      \centering
      \includegraphics[width=\linewidth]{figures/event_start/single/singledis-t.pdf}
      \caption{Verteilung der \(\tau\)}
      \label{fig:multiple_histo_tau}
    \end{minipage}
 \begin{minipage}[t]{0.47\textwidth}
      \centering
      \includegraphics[width=\linewidth]{figures/event_start/single/singledis-Q.pdf}
      \caption{Verteilung der Ladungen}
      \label{fig:multiple_histo_charge}
  \end{minipage}
  \hfill
    \begin{minipage}[t]{0.47\textwidth}
      \centering
      \includegraphics[width=\linewidth]{figures/event_start/single/singledis-R.pdf}
      \caption{Verteilung der \(R\)}
      \label{fig:multiple_histo_R}
    \end{minipage}
 \begin{minipage}[t]{0.47\textwidth}
      \centering
      \includegraphics[width=\linewidth]{figures/event_start/single/singledis-C1.pdf}
      \caption{Verteilung der \(C_1\)}
      \label{fig:multiple_histo_c1}
  \end{minipage}
  \hfill
    \begin{minipage}[t]{0.47\textwidth}
      \centering
      \includegraphics[width=\linewidth]{figures/event_start/single/singledis-C2.pdf}
      \caption{Verteilung der \(C_2\)}
      \label{fig:multiple_histo_c2}
    \end{minipage}
 \begin{minipage}[t]{0.47\textwidth}
      \centering
      \includegraphics[width=\linewidth]{figures/event_start/single/singledis-C3.pdf}
      \caption{Verteilung der \(C_3\)}
      \label{fig:multiple_histo_c3}
  \end{minipage}
\end{figure}













\begin{comment}
  
Der Anstieg des Entladestroms lässt sich durch ein exponentielles Modell beschreiben:
\begin{equation}
    A e^{\frac{t}{\tau}} + C
\end{equation}
Somit lässt sich aus einem solchen fit insbesondere die Anstiegszeit \(\tau\) bestimmen und untersuchen wie sich diese unter verschiedenen Bedingungen verändert. In \figref{fig:start-fit-t} ist die Verteilung der \(\tau\) dargestllt, die sich aus den fits für die verschiedenen Entladungen ergibt. Somit liegen diese in einem Bereich von \SIrange{0,1}{100}{\nano\second} mit Ausreißern, die auf schlechte fits bzw. Ausreißer in den Daten hindeuten. A liegt in dem Interval \SIrange{1}{100}{\micro\ampere} und C in \SIrange{0,1}{100}{\micro\ampere}. In \figref{fig:start-fit-score} ist die Verteilung der scores des fits Verteilt, also wie genau der fit die Daten wiedergibt. Dabei ist zuerkennen, dass der Großteil der Daten sehr gut wiedergegeben wird, es jedoch wenige aber starte Ausreißer gibt, für die der Fit jedoch sehr ungenau ist. Die großem inetwa Normalverteilten Intervale für \(\tau\), A und C entsprechen fits mit einem guten score von über \(0.95\). Aus diesen Intervalen für die Parameter der fits der Entladungen sind mehrer Schlüsse zuziehen. Das C entspricht dem Stromversatz entlang der y-Achse also in dem Fall des Stromanstiegs dem Startstrom \(I_0\). Somit liegen Ströme von einigen \(\mu A\) vor der Entladung vor, diese lassen sich für die höheren Ströme durch die vor der Entladung auftretenden Vorevents wie in \figref{fig:discharge-vor-lang-1} erklären und für die kleineren Ströme können dies Vorevents wie wie in \figref{fig:discharge-vor-mittel-1} sein, die den mittleren Startstrom etwas verschieben und die kleinen \SI{0,1}{\micro\ampere} Ströme lässt sich das durch Rauschen oder auch einem schlecht gewählten fit Interval erklären. In den Abbildungen \ref{fig:start-fit-scatter-gap} bis \ref{fig:start-fit-scatter-u} sind jeweils der Spaltabstand \(d\), der Druck \(p\) und die Durchbruchspannung \(U_B\) gegen \(\tau\) aufgetragen. Es zeigt sich für die \(\tau\) die im Interval mit hohen score liegen, dass die Anstiegszeit nicht linear von diesen Parametern abhängt. Somit sind die Anstiegszeiten uniform über die Paschenkurve verteilt, dies ist ebenfalls in \figref{fig:start-fit-tau-pd} zu sehen.

\begin{figure}[H]
    \centering
    \begin{minipage}[t]{0.32\textwidth}
        \centering
        \includegraphics[width=\textwidth, height=0.2\textheight]{../figures/generated/eventstart/start/stats/Hist-tau-all.pdf}
        \caption{Verteilung der \(\tau\) des fits}
        \label{fig:start-fit-t}
    \end{minipage}
    \hfill
    \begin{minipage}[t]{0.32\textwidth}
        \centering
        \includegraphics[width=\textwidth, height=0.2\textheight]{../figures/generated/eventstart/start/stats/Hist-a-all.pdf}
        \caption{Verteilung der A des fits}
        \label{fig:start-fit-a}
    \end{minipage}
        \hfill
    \begin{minipage}[t]{0.32\textwidth}
        \centering
        \includegraphics[width=\textwidth, height=0.2\textheight]{../figures/generated/eventstart/start/stats/Hist-c-all.pdf}
        \caption{Verteilung der C des fits}
        \label{fig:start-fit-c}
    \end{minipage}
\end{figure}

\begin{figure}[H]
    \centering
    \includegraphics[width=\textwidth]{../figures/generated/eventstart/start/stats/Hist-score-all.pdf}
    \caption{Verteilung des fit scores}
    \label{fig:start-fit-score}
\end{figure}

\begin{figure}[H]
    \centering
    \begin{minipage}[t]{0.32\textwidth}
        \centering
        \includegraphics[width=\textwidth, height=0.2\textheight]{../figures/generated/eventstart/start/stats/scatter-tau-gap.pdf}
        \caption{Der Spaltabstand aufgetragen gegen \(\tau\)}
        \label{fig:start-fit-scatter-gap}
    \end{minipage}
    \hfill
    \begin{minipage}[t]{0.32\textwidth}
        \centering
        \includegraphics[width=\textwidth, height=0.2\textheight]{../figures/generated/eventstart/start/stats/scatter-tau-pressure.pdf}
        \caption{Der Druck aufgetragen gegen \(\tau\)}
        \label{fig:start-fit-scatter-p}
    \end{minipage}
        \hfill
    \begin{minipage}[t]{0.32\textwidth}
        \centering
        \includegraphics[width=\textwidth, height=0.2\textheight]{../figures/generated/eventstart/start/stats/scatter-tau-break.pdf}
        \caption{Die Durchbruchspannung aufgetragen gegen \(\tau\)}
        \label{fig:start-fit-scatter-u}
    \end{minipage}
\end{figure}

\begin{figure}[H]
    \centering
    \includegraphics[width=\textwidth]{../figures/generated/eventstart/start/stats/violin-pd-t_fit.pdf}
    \caption{\(pd\) gegen \(\tau\)}
    \label{fig:start-fit-tau-pd}
\end{figure}

\section{Eventbeginn}
Die verschiedenen Entladungsmechanismen führen zu verschiedenen Eventstrukturen und damit auch zu verschienen Strukturen zu Beginn der Entladung. Für Entladungen mit höherern Energien kommt es zu einem apprupten Anstieg des Stromes über \(\SIrange{1}{5}{\nano\second}\) der in den durchgeführten Messreihen für diese Art der Entladung nicht vollständig aufgelöst werden konnte, da ein Fokus auf die Zeit vor dem Event gelegt wird. Der Exponentielle Charakter entspricht dem Erwarteten einer Elektronenlawine. Für einige der hochenergetischen Entladungen ist in einem Interval von \(\SIrange{0,1}{10}{\micro\second}\) vor dem Event nicht nur der erwartete exponentielle Anstieg zu erkennen, sonders auch \(\SIrange{0.1}{1}{\micro\second}\) lange andauernde Oszillationsmuster zu erkennen. Diese können als die Bildung des noch nicht thermalisierten Streamers interpretiert werden. Für einige dieser Streamer ähnlichen Events entsteht ein Spannungseinbruch zeitgleich zu diesem kleinen Oszillationsmuster. Es werden drei dieser Gruppen beobachtet, Oszillationsmuster die direct in dem exponentiellen Anstieg liegen also direkt mit dem Eventbegin verbunden sind, Muster die einen Abstand von bis zu einigen \(\SI{1}{\micro\second}\) zum Event besitzen und anschließend der Strom wieder auf 0 abfällt und Muster nach denen der Strom leicht erhöht bleibt bei ca. \(\SI{1}{\micro\ampere}\) bis zum eintreten des Events.
\end{comment}

%\section{Vorläuferphenomene Streamer}

