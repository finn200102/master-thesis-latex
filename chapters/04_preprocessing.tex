\chapter{Vorverarbeitung}
\label{chap:preprocessing}

\section{Spannungsteiler-Abweichung}
\label{sec:voltage_divider_diff}
Der in \secref{sec:voltage_divider} behandelte Spannungsteiler führt nicht zu einer 1000:1- Teilung wie erwartet; die am Oszilloskop gemessenen Spannungen sind somit nicht korrekt. In \tabref{tab:voltage-data} sind die Messwerte der Ausgangsspannung des Spannungsteilers in Abhängigkeit von der Eingangsspannung des Netzteils dargestellt. Auf dieser Datengrundlage lässt sich eine lineare Regression durchführen. In \figref{fig:voltage_linear_fit} sind die Messwerte geplottet und die lineare Regression dargestellt. Für die Regression ergibt sich die folgende Formel \(U_{aus} = 1.509\, U_{netz} - 1.236\). Da aus der gemessenen Ausgangsspannung die Netzteilspannung benötigt wird, wird diese Formel nach \(U_{netz}\), also der Netzteilspannung, umgestellt. Die sich so ergebende Formel lautet \(U_{netz} = (U_{aus} + 1.236) / 1.509\). Die gefittete Funktion zeigt eine ausgezeichnete Übereinstimmung mit den Messdaten; dies wird auch durch das Bestimmtheitsmaß \(R^{2} = 0,9986\) belegt. Dies bedeutet, dass über \(99,8\%\) der Streuung durch das lineare Modell erklärt werden kann. Für die weitere Auswertung wird diese Funktion verwendet, um die Spannung zu korrigieren.

\begin{figure}[htbp]
    \centering
    \includegraphics[width=\textwidth]{figures/preprocessing/fit_voltage_data.pdf}
    \caption{Netzteilspannung aufgetragen gegen die Ausgangsspannung mit linearem Fit}
    \label{fig:voltage_linear_fit}
\end{figure}

\begin{table}[H]
\centering
\caption{Messwerte der Spannung}
\label{tab:voltage-data}
\begin{tabular}{S[table-format=2.0] S[table-format=2.1]}
\toprule
\textbf{Netzteilspannung [\si{\kilo\volt}]} & \textbf{Ausgangsspannung [\si{\kilo\volt}]} \\
\midrule
5  & 6.5 \\
6  & 7.5 \\
7  & 9.5 \\
8  & 11.0 \\
9  & 12.5 \\
10 & 13.5 \\
11 & 15.0 \\
12 & 17.0 \\
13 & 18.5 \\
14 & 20.0 \\
\bottomrule
\end{tabular}
\end{table}

\section{Spannungsteiler-Frequenz}
Da die verwendeten Widerstände groß sind, ergibt sich \equref{eq:voltagedividerimp} eine Ausgangsimpedanz von \(Z_{out} = 99{,}90\). Unter Annahme einer parasitären Kapazität von \(C_{in} = \SI{20}{\pico\farad}\) resultiert für die Grenzfrequenz des daraus entstehenden Tiefpassfilters nach \equref{eq: voltagedividerfreq} \(f_c = \SI{80,38}{\kilo\hertz}\).

\section{Zuordnung der fehlenden Spaltendaten}
Für \(2817\) Messungen wurden die Spaltabstände \(d\) nicht erfasst. Diese lassen sich aus dem Zusammenhang zwischen der Durchbruchspannung \(U_B\) und den bekannten Spaltabständen \(d\) unter der Verwendung von k-nearest-neighbors bestimmen. Diese Zuordnung erfolgt für die Druckintervalle \SI{0}{\milli\bar}, \SIrange{0}{1}{\milli\bar}, \SIrange{1}{2}{\milli\bar}, \SIrange{2}{3}{\milli\bar}, \SIrange{3}{4}{\milli\bar}, \SIrange{4}{5}{\milli\bar}, \SIrange{5}{6}{\milli\bar}, \SIrange{6}{7}{\milli\bar} und \SIrange{7}{8}{\milli\bar} jeweils getrennt. In den Abbildungen \ref{fig:fix_gap_p00} bis \ref{fig:fix_gap_p7} ist für diese Druckintervalle der Spaltabstand \(d\) gegen die Spannung \(U_B\) aufgetragen, jeweils für die Messungen, für die der Spaltabstand erfasst wurde, und für diejenigen, für die er zugeordnet wurde. 


\begin{figure}[htbp]
    \centering

    \begin{minipage}[t]{0.32\textwidth}
        \centering
        \includegraphics[width=\linewidth]{../figures/preprocessing/fix_gap__pressure0.pdf}
        \caption{p = \SI{0}{\milli\bar}}
        \label{fig:fix_gap_p00}
    \end{minipage}\hfill
    \begin{minipage}[t]{0.32\textwidth}
        \centering
        \includegraphics[width=\linewidth]{../figures/preprocessing/fix_gap__pressure1.pdf}
        \caption{p = \SIrange{0}{1}{\milli\bar}}
        \label{fig:fix_gap_p0}
    \end{minipage}\hfill
    \begin{minipage}[t]{0.32\textwidth}
        \centering
        \includegraphics[width=\linewidth]{../figures/preprocessing/fix_gap__pressure2.pdf}
        \caption{p = \SIrange{1}{2}{\milli\bar}}
        \label{fig:fix_gap_p1}
    \end{minipage}

    \vspace{0.5em}

    \begin{minipage}[t]{0.32\textwidth}
        \centering
        \includegraphics[width=\linewidth]{../figures/preprocessing/fix_gap__pressure3.pdf}
        \caption{p = \SIrange{2}{3}{\milli\bar}}
        \label{fig:fix_gap_p2}
    \end{minipage}\hfill
    \begin{minipage}[t]{0.32\textwidth}
        \centering
        \includegraphics[width=\linewidth]{../figures/preprocessing/fix_gap__pressure4.pdf}
        \caption{p = \SIrange{3}{4}{\milli\bar}}
        \label{fig:fix_gap_p3}
    \end{minipage}\hfill
    \begin{minipage}[t]{0.32\textwidth}
        \centering
        \includegraphics[width=\linewidth]{../figures/preprocessing/fix_gap__pressure5.pdf}
        \caption{p = \SIrange{4}{5}{\milli\bar}}
        \label{fig:fix_gap_p4}
    \end{minipage}

    \vspace{0.5em}

    \begin{minipage}[t]{0.32\textwidth}
        \centering
        \includegraphics[width=\linewidth]{../figures/preprocessing/fix_gap__pressure6.pdf}
        \caption{p = \SIrange{5}{6}{\milli\bar}}
        \label{fig:fix_gap_p5}
    \end{minipage}\hfill
    \begin{minipage}[t]{0.32\textwidth}
        \centering
        \includegraphics[width=\linewidth]{../figures/preprocessing/fix_gap__pressure7.pdf}
        \caption{p = \SIrange{6}{7}{\milli\bar}}
        \label{fig:fix_gap_p6}
    \end{minipage}\hfill
    \begin{minipage}[t]{0.32\textwidth}
        \centering
        \includegraphics[width=\linewidth]{../figures/preprocessing/fix_gap__pressure8.pdf}
        \caption{p = \SIrange{7}{8}{\milli\bar}}
        \label{fig:fix_gap_p7}
    \end{minipage}

    \caption{Spannung \( U \) gegen Spaltabstand \( d \) für verschiedene Drücke \( p \).}
    \label{fig:fix_gap_all}
\end{figure}

  
\section{Zuordnung der fehlenden Druckdaten}
\label{sec:setpressure}
Für \(5524\) Messungen wurden die Drücke \(p\) nicht erfasst. Diese lassen sich aus dem Zusammenhang zwischen der Durchbruchspannung \(U_B\) und dem Druck \(p\) unter Verwendung des k-nearest-neighbors-Verfahrens bestimmen. Die Zuordnung erfolgt für die einzelnen Spaltabstände getrennt. In den Abbildungen \ref{fig:fix_pressure_g1} bis \ref{fig:fix_pressure_g15} ist für diese Spaltabstände der Druck \(p\) gegen die Durchbruchspannung \(U_B\) aufgetragen. 


\begin{figure}[p] % use [p] or [htbp] to allow full-page float
    \centering

    \begin{minipage}[t]{0.31\textwidth}
        \centering
        \includegraphics[width=\linewidth]{../figures/preprocessing/fix_pressure__gap1.0.pdf}
        \subcaption{\( d = \SI{1}{\milli\meter} \)}
        \label{fig:fix_pressure_g1}
    \end{minipage}\hfill
    \begin{minipage}[t]{0.31\textwidth}
        \centering
        \includegraphics[width=\linewidth]{../figures/preprocessing/fix_pressure__gap3.0.pdf}
        \subcaption{\( d = \SI{3}{\milli\meter} \)}
        \label{fig:fix_pressure_g3}
    \end{minipage}\hfill
    \begin{minipage}[t]{0.31\textwidth}
        \centering
        \includegraphics[width=\linewidth]{../figures/preprocessing/fix_pressure__gap4.0.pdf}
        \subcaption{\( d = \SI{4}{\milli\meter} \)}
        \label{fig:fix_pressure_g4}
    \end{minipage}

    \vspace{0.5em}

    \begin{minipage}[t]{0.31\textwidth}
        \centering
        \includegraphics[width=\linewidth]{../figures/preprocessing/fix_pressure__gap6.0.pdf}
        \subcaption{\( d = \SI{6}{\milli\meter} \)}
        \label{fig:fix_pressure_g6}
    \end{minipage}\hfill
    \begin{minipage}[t]{0.31\textwidth}
        \centering
        \includegraphics[width=\linewidth]{../figures/preprocessing/fix_pressure__gap7.0.pdf}
        \subcaption{\( d = \SI{7}{\milli\meter} \)}
        \label{fig:fix_pressure_g7}
    \end{minipage}\hfill
    \begin{minipage}[t]{0.31\textwidth}
        \centering
        \includegraphics[width=\linewidth]{../figures/preprocessing/fix_pressure__gap8.0.pdf}
        \subcaption{\( d = \SI{8}{\milli\meter} \)}
        \label{fig:fix_pressure_g8}
    \end{minipage}

    \vspace{0.5em}

    \begin{minipage}[t]{0.31\textwidth}
        \centering
        \includegraphics[width=\linewidth]{../figures/preprocessing/fix_pressure__gap10.0.pdf}
        \subcaption{\( d = \SI{10}{\milli\meter} \)}
        \label{fig:fix_pressure_g10}
    \end{minipage}\hfill
    \begin{minipage}[t]{0.31\textwidth}
        \centering
        \includegraphics[width=\linewidth]{../figures/preprocessing/fix_pressure__gap11.0.pdf}
        \subcaption{\( d = \SI{11}{\milli\meter} \)}
        \label{fig:fix_pressure_g11}
    \end{minipage}\hfill
    \begin{minipage}[t]{0.31\textwidth}
        \centering
        \includegraphics[width=\linewidth]{../figures/preprocessing/fix_pressure__gap12.0.pdf}
        \subcaption{\( d = \SI{12}{\milli\meter} \)}
        \label{fig:fix_pressure_g12}
    \end{minipage}

    \vspace{0.5em}

    \begin{minipage}[t]{0.31\textwidth}
        \centering
        \includegraphics[width=\linewidth]{../figures/preprocessing/fix_pressure__gap13.0.pdf}
        \subcaption{\( d = \SI{13}{\milli\meter} \)}
        \label{fig:fix_pressure_g13}
    \end{minipage}\hfill
    \begin{minipage}[t]{0.31\textwidth}
        \centering
        \includegraphics[width=\linewidth]{../figures/preprocessing/fix_pressure__gap15.0.pdf}
        \subcaption{\( d = \SI{15}{\milli\meter} \)}
        \label{fig:fix_pressure_g15}
    \end{minipage}

    \caption{Spannung \( U \) gegen Druck \( p \) für verschiedene Spaltabstände \( d \).}
    \label{fig:fix_pressure_all}
\end{figure}
