\chapter{Vorverarbeitung}
\label{chap:preprocessing}

\section{Spannungsteiler Abweichung}
\label{sec:voltage_divider_diff}
Der in \secref{sec:voltage_divider} behandelte Spannungsteiler führt nicht zu einer 1000:1 Teilung wie erwartet, die gemessenen Spannungen am Oszilloskop sind somit nicht korrekt. In \tabref{tab:voltage-data} sind die Messwerte der Ausgangsspannung des Spannungsteilers in Abhängigkeit von der Eingangsspannung des Netzteils dargestellt. Auf diesen Daten lässt sich eine Lineareregression durchführen. In \figref{fig:voltage_linear_fit} sind die Messwerte geplottet und die lineare Regression dargestellt. Für die Regression ergibt sich die folgene Formel $U_{aus} = 1.509 U_{netz} - 1.236$. Da aus der gemessenen Ausgangsspannung die Netzteil-Spannung benötigt wird wird diese Formel noch nach $U_{netz}$, also der Netzteil-Spannung, umgestellt.  Die sich so ergebene Formel ist: $U_{netz} = (U_{aus} + 1.236) / 1.509$. Die gefittete Funktion zeigt eine ausgezeichnete Übereinstimmung mit den Messdaten, dies wird auch durch das Bestimmtheitsmaß $R^{2} = 0,9986$ belegt. Dies bedeutet, dass über $99,8\%$ der Streuung druch das lineare Modell erklärt werden kann. Für die weitere Auswertung wird diese Funktion verwendet um die Spannung zu korigieren.

\begin{figure}[htbp]
    \centering
    \includegraphics[width=\textwidth]{figures/preprocessing/fit_voltage_data.pdf}
    \caption{Netzteil-Spannung aufgetragen gegen die Ausgangsspannung mit linearen fit}
    \label{fig:voltage_linear_fit}
\end{figure}

\begin{table}[H]
\centering
\caption{Messwerte der Spannung}
\label{tab:voltage-data}
\begin{tabular}{S[table-format=2.0] S[table-format=2.1]}
\toprule
\textbf{Netzteil-Spannung [\si{\kilo\volt}]} & \textbf{Ausgangsspannung [\si{\kilo\volt}]} \\
\midrule
5  & 6.5 \\
6  & 7.5 \\
7  & 9.5 \\
8  & 11.0 \\
9  & 12.5 \\
10 & 13.5 \\
11 & 15.0 \\
12 & 17.0 \\
13 & 18.5 \\
14 & 20.0 \\
\bottomrule
\end{tabular}
\end{table}

\section{Spannungsteiler Frequenz}
Da die verwenden Widerstände groß sind, folgt nach \equref{eq:voltagedividerimp} eine Ausgangsimpendanz von \(Z_{out} = 99,90\) und mit einer angenommenen parasitären Kapazitivität von \(C_{in} = \SI{20}{\pico\farad}\) ergibt sich für die Frequenz des sich ergebenden Tiefpassfilters nach \equref{eq: voltagedividerfreq} \(f_c = \SI{80,38}{\kilo\hertz}\).

\section{Zuordnung der fehlenden Spaltdaten}
  \begin{figure}[htbp]
    \centering
    % Top row
    \begin{minipage}[t]{0.3\textwidth}
        \centering
        \includegraphics[width=\textwidth]{../figures/preprocessing/fix_gap__pressure0.pdf}
        \caption{Spannung U gegen Spaltabstand d für den Druck p = \SI{0}{\milli\bar}}
        \label{fig:fix_gap_p00}
    \end{minipage}\hfill

    \begin{minipage}[t]{0.3\textwidth}
        \centering
        \includegraphics[width=\textwidth]{../figures/preprocessing/fix_gap__pressure1.pdf}
        \caption{Spannung U gegen Spaltabstand d für den Druck p = \SIrange{0}{1}{\milli\bar}}
        \label{fig:fix_gap_p0}
    \end{minipage}\hfill
    \begin{minipage}[t]{0.3\textwidth}
        \centering
        \includegraphics[width=\textwidth]{../figures/preprocessing/fix_gap__pressure2.pdf}
        \caption{Spannung U gegen Spaltabstand d für den Druck p = \SIrange{1}{2}{\milli\bar}}
        \label{fig:fix_gap_p1}
    \end{minipage}\hfill
    \begin{minipage}[t]{0.3\textwidth}
        \centering
        \includegraphics[width=\textwidth]{../figures/preprocessing/fix_gap__pressure3.pdf}
        \caption{Spannung U gegen Spaltabstand d für den Druck p = \SIrange{2}{3}{\milli\bar}}
        \label{fig:fix_gap_p2}
    \end{minipage}

    \begin{minipage}[t]{0.3\textwidth}
        \centering
        \includegraphics[width=\textwidth]{../figures/preprocessing/fix_gap__pressure4.pdf}
        \caption{Spannung U gegen Spaltabstand d für den Druck p = \SIrange{3}{4}{\milli\bar}}
        \label{fig:fix_gap_p3}
    \end{minipage}\hfill
    \begin{minipage}[t]{0.3\textwidth}
        \centering
        \includegraphics[width=\textwidth]{../figures/preprocessing/fix_gap__pressure5.pdf}
        \caption{Spannung U gegen Spaltabstand d für den Druck p = \SIrange{4}{5}{\milli\bar}}
        \label{fig:fix_gap_p4}
    \end{minipage}\hfill
    \begin{minipage}[t]{0.3\textwidth}
        \centering
        \includegraphics[width=\textwidth]{../figures/preprocessing/fix_gap__pressure6.pdf}
        \caption{Spannung U gegen Spaltabstand d für den Druck p = \SIrange{5}{6}{\milli\bar}}
        \label{fig:fix_gap_p5}
    \end{minipage}

    \begin{minipage}[t]{0.3\textwidth}
        \centering
        \includegraphics[width=\textwidth]{../figures/preprocessing/fix_gap__pressure7.pdf}
        \caption{Spannung U gegen Spaltabstand d für den Druck p = \SIrange{6}{7}{\milli\bar}}
        \label{fig:fix_gap_p6}
    \end{minipage}\hfill
    \begin{minipage}[t]{0.3\textwidth}
        \centering
        \includegraphics[width=\textwidth]{../figures/preprocessing/fix_gap__pressure8.pdf}
        \caption{Spannung U gegen Spaltabstand d für den Druck p = \SIrange{7}{8}{\milli\bar}}
        \label{fig:fix_gap_p7}
    \end{minipage}\hfill



    \vspace{0.5em} % vertical spacing between rows
   \end{figure}



\section{Zuordnung von p = 0}
\label{sec:pressurenull}

%Wie in \figref{fig:histogram-pressure-stab} und \figref{fig:histogram-pressure-platte} zu sehen liegen etwa \(3780\) Events vor, beidenen ein Druck von 0 aufgezeichnet wurde, dies ist nicht physikalisch und auf Kommunikationsprobleme zwischen der Druckmessröhre und dem Server zurückzuführen. Um diese Daten für die weitere Auswertung verwenden zu können werden diese ihren in der Spannung am nächsten liegenden Punkten zugeordnet. Für 5 Spaltabstände liegen solche Drücke vor, dies deutet ebenfalls darauf hin, das dies ein lokales Problem ist und nur für diese Messreihen aufgetreten ist. In \tabref{tab:updated-discharges} sind die beteiligten Spaltabstände und die Anzahl der fehlerhaften Daten für diese dargestellt. In \figref{fig:fixp-gap-1.0} bis \ref{fig:fixp-gap-11.0} sind die Plots für das jeweilige Verhältnis aus Spannung und Druck aufgetragen. Für \ref{fig:fixp-gap-3.0} bis \ref{fig:fixp-gap-11.0} werden zwei fits benötigt, ein quadratischer für Spannungen unter \(\SI{6}{\kilo\volt}\) und ein linearer Fit darüber.

\begin{table}[H]
\centering
\caption{Anzahl der Entladungen mit \(p = \SI{0}{\milli\bar}\) für verschiedene Spaltabstände}
\label{tab:updated-discharges}
\begin{tabular}{S[table-format=2.1] S[table-format=3.0]}
\toprule
\textbf{Spaltabstand \(d\) [\si{\milli\meter}]} & \textbf{Anzahl Entladungen} \\
\midrule
1.0  & 510 \\
3.0  & 75  \\
6.0  & 600 \\
7.0  & 14  \\
11.0 & 680 \\
\bottomrule
\end{tabular}
\end{table}

\begin{figure}[htbp]
    \centering
    % Top row
    \begin{minipage}[t]{0.3\textwidth}
        \centering
        \includegraphics[width=\textwidth]{../figures/preprocessing/fix_pressure__gap1.0.pdf}
        \caption{Spannung U gegen Druck p für den Spaltabstand d = \SI{1}{\milli\meter}}
        \label{fig:fix_pressure_g1}
    \end{minipage}\hfill
     \begin{minipage}[t]{0.3\textwidth}
        \centering
        \includegraphics[width=\textwidth]{../figures/preprocessing/fix_pressure__gap3.0.pdf}
        \caption{Spannung U gegen Druck p für den Spaltabstand d = \SI{3}{\milli\meter}}
        \label{fig:fix_pressure_g3}
    \end{minipage}\hfill

   \begin{minipage}[t]{0.3\textwidth}
        \centering
        \includegraphics[width=\textwidth]{../figures/preprocessing/fix_pressure__gap4.0.pdf}
        \caption{Spannung U gegen Druck p für den Spaltabstand d = \SI{4}{\milli\meter}}
        \label{fig:fix_pressure_g4}
    \end{minipage}\hfill

   \begin{minipage}[t]{0.3\textwidth}
        \centering
        \includegraphics[width=\textwidth]{../figures/preprocessing/fix_pressure__gap6.0.pdf}
        \caption{Spannung U gegen Druck p für den Spaltabstand d = \SI{6}{\milli\meter}}
        \label{fig:fix_pressure_g6}
    \end{minipage}\hfill

   \begin{minipage}[t]{0.3\textwidth}
        \centering
        \includegraphics[width=\textwidth]{../figures/preprocessing/fix_pressure__gap7.0.pdf}
        \caption{Spannung U gegen Druck p für den Spaltabstand d = \SI{7}{\milli\meter}}
        \label{fig:fix_pressure_g7}
    \end{minipage}\hfill

   \begin{minipage}[t]{0.3\textwidth}
        \centering
        \includegraphics[width=\textwidth]{../figures/preprocessing/fix_pressure__gap8.0.pdf}
        \caption{Spannung U gegen Druck p für den Spaltabstand d = \SI{8}{\milli\meter}}
        \label{fig:fix_pressure_g8}
    \end{minipage}\hfill

   \begin{minipage}[t]{0.3\textwidth}
        \centering
        \includegraphics[width=\textwidth]{../figures/preprocessing/fix_pressure__gap10.0.pdf}
        \caption{Spannung U gegen Druck p für den Spaltabstand d = \SI{10}{\milli\meter}}
        \label{fig:fix_pressure_g10}
    \end{minipage}\hfill

   \begin{minipage}[t]{0.3\textwidth}
        \centering
        \includegraphics[width=\textwidth]{../figures/preprocessing/fix_pressure__gap11.0.pdf}
        \caption{Spannung U gegen Druck p für den Spaltabstand d = \SI{11}{\milli\meter}}
        \label{fig:fix_pressure_g11}
    \end{minipage}\hfill

   \begin{minipage}[t]{0.3\textwidth}
        \centering
        \includegraphics[width=\textwidth]{../figures/preprocessing/fix_pressure__gap12.0.pdf}
        \caption{Spannung U gegen Druck p für den Spaltabstand d = \SI{12}{\milli\meter}}
        \label{fig:fix_pressure_g12}
    \end{minipage}\hfill
   \begin{minipage}[t]{0.3\textwidth}
        \centering
        \includegraphics[width=\textwidth]{../figures/preprocessing/fix_pressure__gap13.0.pdf}
        \caption{Spannung U gegen Druck p für den Spaltabstand d = \SI{13}{\milli\meter}}
        \label{fig:fix_pressure_g13}
    \end{minipage}\hfill

   \begin{minipage}[t]{0.3\textwidth}
        \centering
        \includegraphics[width=\textwidth]{../figures/preprocessing/fix_pressure__gap15.0.pdf}
        \caption{Spannung U gegen Druck p für den Spaltabstand d = \SI{15}{\milli\meter}}
        \label{fig:fix_pressure_g15}
    \end{minipage}\hfill






    \vspace{0.5em} % vertical spacing between rows
   \end{figure}




\section{Zuordnung von p > 7}
\label{sec:pressurelarge}
%Die verwendete Messröhre kann nur Daten bis \(\SI{7,2}{\milli\bar}\) aufzeichnen, jedoch wurden auch Experimente in einem höheren Druckbereich von ca. \(\SI{700}{\milli\bar}\) durchgeführt. In den Für \ref{fig:fixp-gap-3.0} bis \ref{fig:fixp-gap-11.0} ist ein Abflachen des Verhältnises aus Spannung und Druck ab etwa \(\SI{7}{\milli\bar}\) zu erkennen, dieser Wert deckt sich mit der Erfahrung aus der Betrachtung der Daten. Somit werden Drücke über dieser Schwelle im folgenden als hohe Drücke bezeichnet.




\section{Entfernung von Nan}
\label{sec:nan_removal}
Für Messreihen für die der Triggerpunk sehr niedrig gesetzt wurde, ergeben sich Noise Werte die die anderen in dieser Dimension aufgenommenen Daten verunreinigen, die Werte lassen sich einfach filter, da nur für diese der Strom so niedrig ist unter den Messreihen indenen Entladungen aufgenommen wurden. In \figref{fig:noise} ist ein Beispiel für Noise im Strom geplottet.

\begin{figure}[htbp]
    \centering
    \includegraphics[width=\textwidth]{../figures/preprocessing/discharge_noise.pdf}
    \caption{Noise für Ströme unter \(\SI{1}{\micro\ampere}\)}
    \label{fig:noise}
\end{figure}


