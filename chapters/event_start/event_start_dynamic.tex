\section{Eventstartdynamik}


Die Struktur der vorliegenen Vorläuferphenomene ist im weiteren interessant. In \figref{fig:discharge-vor-klein-1} und in \figref{fig:discharge-vor-klein-2} sind zwei erkannte Vorläuferphenomene für den Bereich \(\SIrange{10}{100}{\nano\second}\) dargestellt, hierbei ist zu erkennen, dass einige (signifikanter Anteil) der erkannten Vorläufer im wesentlichen nur dem exponentiellen Anstieg zu Beginn des Events entsprechen, diese Klasse ist aber trotzdem im Hinblick auf den Eventstart interessant, da hier größere Anstiegszeiten auftreten, da die sonstigen apprupten Anstiege die schneller als \(\SI{10}{\nano\second}\) sind von vorhinein für die Betrachtung von Vorläufern herausgefiltert wurden. In \figref{fig:discharge-vor-klein-2} sind jedoch im Anstieg kleine Oszillationen zuerkennen, also Phänomene die charakteristisch für das Event sein sollten. In den Abbildungen \ref{fig:discharge-vor-mittel-1} bis \ref{fig:discharge-vor-mittel-2} sind zwei Entladungen exemplarisch für den mittleren Bereich von \(\SIrange{0,1}{1}{\micro\second}\) dargestellt. In diesem Bereich treten längere Vorläuferprozesse auf, die analog wie auch in \figref{fig:discharge-vor-klein-2} direkt vor dem Event stattfinden, entweder wie in \figref{fig:discharge-vor-mittel-2} den exponentiellen Anstieg begleiten oder wie in \figref{fig:discharge-vor-mittel-2} wo die Oszillationen eine höhere zweitliche Ausdehnung, höhere Amplituden und schon vor dem exponentiellen Anstiegt stattfinden, also nicht mehr nur begleitend auftreten. In \figref{fig:discharge-vor-lang-1} und \figref{fig:discharge-vor-lang-2} sind zwei Entladungen aus dem letzten Interval für \(s > \SI{1}{\micro\second}\) dargestellt. Neben der in Entladung in \figref{fig:discharge-vor-lang-2} die ähnliche wie \figref{fig:discharge-vor-mittel-1} vor dem Event stattfindet, aber trotzdem direkt im exponentiellen Anstieg des Events endet, ist in \figref{fig:discharge-vor-lang-2} ein weitere Typ and Vorläufer zu erkennen, zumeinem ist der Start diese Vorläufers noch weiter vom Start des Event entfernt, aber der oszillierende Anteil regt sich wieder ab und es bleibt zwischen Vorläufer und Event ein konstanter kleiner Strom stehen, bis es zum Event kommt. Dieser Prozess lässt sich als streamerartig identifizieren, wobei der anfängliche oszillierende Teil sich als die Lawine und die Bildung der Streamerkanäle interpretieren lässt, kann der konstante fließende Strom als der Übergang zwischen Streamer hin zum Leader und der Sparkentladung interpretieren, also der Thermalisierung diese anfänglich nur schwach leitenden Kanals. Gerade diese Vorläufersignaturen die durch einen größeren Abstand zum Event charakterisiert sind und dier erste Bunch an Oszillationen ein Ende aufweist, nach dem dieser Strom konstant bis hin zum Start der Entladungen ist, eignen sich um zu untersuchen inwie fern sich diese Vorläufersignaturen eignen um den Vorläuferabstand zu bestimmen. 

\begin{figure}[H]
    \centering
    \begin{minipage}[t]{0.44\textwidth}
        \centering
        \includegraphics[width=\textwidth, height=0.25\textheight]{../figures/event_start/discharge_short_1.pdf}
        \caption{Entladung für kleine Vorläuferabstände}
        \label{fig:discharge-vor-klein-1}
    \end{minipage}
    \hfill
    \begin{minipage}[t]{0.44\textwidth}
        \centering
        \includegraphics[width=\textwidth, height=0.25\textheight]{../figures/event_start/discharge_short_2.pdf}
        \caption{Entladungen für kleine Vorläuferabstände}
        \label{fig:discharge-vor-klein-2}
    \end{minipage}
\end{figure}



\begin{figure}[H]
    \centering
    \begin{minipage}[t]{0.44\textwidth}
        \centering
        \includegraphics[width=\textwidth, height=0.25\textheight]{../figures/event_start/discharge_middle_1.pdf}
        \caption{Entladung für mittlere Vorläuferabstände}
        \label{fig:discharge-vor-mittel-1}
    \end{minipage}
    \hfill
    \begin{minipage}[t]{0.44\textwidth}
        \centering
        \includegraphics[width=\textwidth, height=0.25\textheight]{../figures/event_start/discharge_middle_2.pdf}
        \caption{Entladungen für mittlere Vorläuferabstände}
        \label{fig:discharge-vor-mittel-2}
    \end{minipage}
\end{figure}

\begin{figure}[H]
    \centering
    \begin{minipage}[t]{0.44\textwidth}
        \centering
        \includegraphics[width=\textwidth, height=0.25\textheight]{../figures/event_start/discharge_long_1.pdf}
        \caption{Entladung für längere Vorläuferabstände}
        \label{fig:discharge-vor-lang-1}
    \end{minipage}
    \hfill
    \begin{minipage}[t]{0.44\textwidth}
        \centering
        \includegraphics[width=\textwidth, height=0.25\textheight]{../figures/event_start/discharge_long_2.pdf}
        \caption{Entladungen für längere Vorläuferabstände}
        \label{fig:discharge-vor-lang-2}
    \end{minipage}
\end{figure}

