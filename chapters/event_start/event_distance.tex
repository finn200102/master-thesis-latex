\section{Eventabstände}
\label{sec:eventdistance}

Der in \secref{sec:event_distance} definierte Abstand ist für die Entladungen nach \figref{fig:highres-discharge} in \figref{fig:histogram-eventdistance} für die automatische Definition der Eventgrenzen und in \figref{fig:histogram-eventdistance-hand} für die händisch gesetzten Grenzen dargestellt. In \ref{fig:histogram-eventdistance} sind vier Bereiche ersichtlich, zwei die an den Grenzen liegen, bei \SI{1e-9}{\second} und bei \SI{1e-5}{\second}. Diese beiden Regionen entsprechen für den kleinen Eventabstände garnicht bzw. kaum vorhandenen Vorevents, also dem Fall, dass Eventstart gleich Vorläuferstart ist und den großen Eventabständen bei denen die automatische Bestimmung der Vorläuferstartes zu sensitiv auf frühe Schwankungen ist. Die beiden mittleren Intervalle von \SIrange{2e-9}{3e-8}{\second} und von \SIrange{3e-8}{1e-5}{\second}. Das erste Intervall entspricht kurzen Vorläuferphänomenen, bei denen die meißten direkt am Eventstart hängen und entsprechen für die kürzesten Oszillationen, die direkt auf dem exponentiellen Anstieg des Events liegen und für die etwas längeren Oszillationen, die noch kurz vor dem exponentiellen Eventstart liegen. Im zweiten Intervall liegen sowohl weitere flache Oszillationen die wie im ersten Intervall direkt vor dem Event liegen aber auch Oszillationen mit einem stärkeren Peak nach dem ein konstanter kleiner Strom fließt, dieser hat Streamercharakter. Der Vergleich der automatisch gesetzten Grenzen mit den händischen in \figref{fig:histogram-eventdistance-hand} zeigt zum einen, dass weniger Entladungen händisch annotiert wurden, aber auch dass nur eine Verteilung vorliegt, die einen normalverteilten Charakter aufweist, mit einen Schweif nach rechts. Diese Verteilung entspricht dem dritten Bereich der automatisch gesetzen Grenzen, da für die händische Annotation nur die längeren besonders Interessanten Vorevents gewählt wurden. In den Abbildungen \ref{fig:box-eventdistance-electrode} bis \ref{fig:box-eventdistance-plate-hand} sind die Eventabstände für die beiden Elektrodengeometrien und für die aautomatische und händische Version aufgetragen gegen den Spaltabstand. Es ist zubeachten, dass die y-Axis der Plots also der Eventabstand logarithmisch skalliert ist. Somit entspricht der klare lineare Trend einem exponentiellen Trend von kleinen Spaltabständen hin zu großen Spaltabständen für den Eventabstand. Aufgrund der geringeren Datenmenge und der selektiven Auswahl dieser ist dieser Trend weniger gut erkennbar und schwächer ausgeprägt für die händisch gesetzten Grenzen, jedoch in \figref{fig:box-eventdistance-plate-hand} trotzdem erkennbar.







\begin{figure}[htbp]
    \centering
    \begin{minipage}[t]{0.47\textwidth}
      \centering
      \includegraphics[width=\linewidth]{../figures/event_start/event_distance.pdf}
      \caption{Verteilung der Eventabstände}
      \label{fig:histogram-eventdistance}
    \end{minipage}
 \begin{minipage}[t]{0.47\textwidth}
      \centering
      \includegraphics[width=\linewidth]{../figures/event_start/event_distance_hand.pdf}
      \caption{Verteilung der Eventabstände (händisch)}
      \label{fig:histogram-eventdistance-hand}
  \end{minipage}
\end{figure}

\begin{figure}[htbp]
    \centering
    \begin{minipage}[t]{0.47\textwidth}
      \centering
      \includegraphics[width=\linewidth]{../figures/event_start/event_distance_box_electrodes.pdf}
      \caption{Spaltabstand d gegen Eventabstände (Elektrode)}
      \label{fig:box-eventdistance-electrode}
   \end{minipage}
 \begin{minipage}[t]{0.47\textwidth}
      \centering
      \includegraphics[width=\linewidth]{../figures/event_start/event_distance_box_plates.pdf}
      \caption{Spaltabstand d gegen Eventabstände (Platte)}
      \label{fig:box-eventdistance-plate}
  \end{minipage}
\begin{minipage}[t]{0.47\textwidth}
      \centering
      \includegraphics[width=\linewidth]{../figures/event_start/event_distance_manual_box_electrodes.pdf}
      \caption{Spaltabstand d gegen Eventabstände (Elektrode, händisch)}
      \label{fig:box-eventdistance-electrode-hand}
   \end{minipage}
 \begin{minipage}[t]{0.47\textwidth}
      \centering
      \includegraphics[width=\linewidth]{../figures/event_start/event_distance_manual_box_plates.pdf}
      \caption{Spaltabstand d gegen Eventabstände (Platte, händisch)}
      \label{fig:box-eventdistance-plate-hand}
  \end{minipage}

\end{figure}

\subsection{Vorläuferabstände}
\label{sec:precursor_distance_target}
Die Betrachtung von \figref{fig:histogram-precdistance} und der Vergleich mit \figref{fig:histogram-eventdistance} zeigt, dass sowohl die beiden Grenzbereiche nicht mehr vorliegen. Der linke fällt weg, da kein Vorläuferende für solch kurze Phänomene auftritt und das rechte fällt weg, da diese großen Abstände Fehlern in dem Grenzsetzung entsprechen. Ebenfall ist erkenntlich, dass nur einem deutlich kleinerer Teil der Entladungen eine Ende zugeordnet werden konnte. Und dass diese Reduktion der Daten für die beiden Intervalle unterschiedlich stark erfolgte und somit das Verhältnis dieser beiden Bereich ausgetauscht hat.


\begin{figure}[htbp]
    \centering
    \begin{minipage}[t]{0.47\textwidth}
      \centering
      \includegraphics[width=\linewidth]{../figures/event_start/precursor_distance.pdf}
      \caption{Verteilung der Vorläuferabstände}
      \label{fig:histogram-precdistance}
   \end{minipage}
 \begin{minipage}[t]{0.47\textwidth}
      \centering
      \includegraphics[width=\linewidth]{../figures/event_start/precursor_distance_hand.pdf}
      \caption{Verteilung der Vorläuferabstände (händisch)}
      \label{fig:histogram-precdistance-hand}
  \end{minipage}
\end{figure}

\begin{figure}[htbp]
    \centering
    \begin{minipage}[t]{0.47\textwidth}
      \centering
      \includegraphics[width=\linewidth]{../figures/event_start/precursor_distance_box_electrodes.pdf}
      \caption{Spaltabstand d gegen Vorläuferabstände (Elektrode)}
      \label{fig:box-precdistance-electrode}
   \end{minipage}
 \begin{minipage}[t]{0.47\textwidth}
      \centering
      \includegraphics[width=\linewidth]{../figures/event_start/precursor_distance_box_plates.pdf}
      \caption{Spaltabstand d gegen Vorläuferabstände (Platte)}
      \label{fig:box-precdistance-plate}
  \end{minipage}
\begin{minipage}[t]{0.47\textwidth}
      \centering
      \includegraphics[width=\linewidth]{../figures/event_start/precursor_distance_manual_box_electrodes.pdf}
      \caption{Spaltabstand d gegen Vorläuferabstände (Elektrode, händisch)}
      \label{fig:box-precdistance-electrode-hand}
   \end{minipage}
 \begin{minipage}[t]{0.47\textwidth}
      \centering
      \includegraphics[width=\linewidth]{../figures/event_start/precursor_distance_manual_box_plates.pdf}
      \caption{Spaltabstand d gegen Vorläuferabstände (Platte, händisch)}
      \label{fig:box-precdistance-plate-hand}
  \end{minipage}

\end{figure}

