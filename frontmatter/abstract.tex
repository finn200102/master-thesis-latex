\chapter*{Abstract}
\addcontentsline{toc}{chapter}{Abstract}

Physikalische Systeme neigen dazu, einen Gleichgewichtszustand anzustreben. Dennoch existieren scheinbar stablie Zustände, die auf kurzen Zeitskalen als konstant erscheinen, jedoch plötzlich in energetisch günstigere übergehen oder vollständig kollabieren. Solche Übergänge treten in verschiedenen natürlichen oder auch technischen Systemen auf, in epileptischen Anfällen, bei Erdbeben oder auch bei elektrischen Entladungen. In dieser Arbeit wird eine Funkenstreche als experimentelles Modell verwendet, um den Charakter und die zeitliche Dynamik solcher Systeme zu analysieren. Es wurden Strom und Spannungsverläufe unter verschiedenen Bedingungen aufgezeichnet. Besonderer Fokus dieser Arbeit liegt auf der Identifikation von Vorläuferphänomenen wie Streamern und der Untersuchung der Verwendung dieser um den Zusammenbruch vorherzusagen. 
